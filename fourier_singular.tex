\documentclass[11pt]{article}

\usepackage[utf8]{inputenc}
\usepackage{amsmath, amsfonts, amssymb, amsthm, bbm, cases, csquotes, enumitem, hyperref, mathrsfs, mathtools, tikz, tikz-cd}
\usetikzlibrary{calc}
\usetikzlibrary{quotes,angles}
\usepackage[style=alphabetic]{biblatex}

\addbibresource{fourier_singular.bib}

\author{Petr Kosenko}
\title{On a complex-analytic approach to classifying stationary measures on $S^1$ with respect to the countably supported measures on $PSU(1,1)$}
%\copyrighttext{text}

\newtheorem{definition}{Definition}[section]
\newtheorem{theorem}{Theorem}[section]
\newtheorem{proposition}{Proposition}[section]
\newtheorem{corollary}{Corollary}[section]
\newtheorem{lemma}{Lemma}[section]
\newtheorem{example}{Example}[section]
\newtheorem{conjecture}{Conjecture}[section]

\usepackage[left=2cm,right=2cm,top=2cm,bottom=2cm,bindingoffset=0cm]{geometry}

\newcommand{\Addresses}{{% additional braces for segregating \footnotesize
		\bigskip
		\footnotesize
		\noindent
		\textbf{Petr Kosenko}, \\
		Department of Mathematics, University of British Columbia, 1984 Mathematics Road, Vancouver, BC, Canada, V6T 1Z2 \\	
		\textit{E-mail:} \texttt{pkosenko@math.ubc.ca} \\
		\textit{ORCID ID:} https://orcid.org/0000-0002-4150-0613
}}


\begin{document}

\maketitle

\begin{abstract}
We provide a complex-analytic approach to the classification of stationary probability measures on $S^1$ with respect to complex finite Borel measures on $PSU(1,1)$ satisfying the finite first moment condition by studying their Poisson-like and Cauchy-Szeg\H{o} transforms from the perspective of generalized analytic continuation. This approach allows us to establish the singularity of the hitting measure for finitely supported admissible random walks on cocompact lattices in $PSU(1,1)$, affirmatively resolving the Kaimanovich-le Prince's singularity conjecture. In addition, it suggests alternative proofs to the several well-known related results by Furstenberg, Kaimanovich, Guivarc'h-le Jan and Bourgain, among several others.
\end{abstract}

\section{Introduction}

\subsection{Background}

One of the most important notions in dynamical systems is of an \textbf{invariant measure}: given a topological space $X$ and a self-map $T : X \rightarrow X$, one can study Borel measures $\mu \in Bor(X)$ on $X$ which satisfy
\begin{equation}
	(T_* \mu)(A) = \mu(T^{-1}(A)) = \mu(A).
\end{equation}

This notion works quite well when provided with a single map $X \rightarrow X$. However, one often encounters nice spaces equipped with group actions $\Gamma \curvearrowright X$, and it is entirely possible that there are no measures invariant with respect to every element $\gamma \in \Gamma$.

Nevertheless, there is a natural weakening of the above definition, requiring a measure to be invariant ``on average''.

\begin{definition}
	Consider a group action $\Gamma \curvearrowright X$. Let $\nu$ be a Borel probability measure on $X$. Given a Borel measure $\mu$ on $\Gamma$, we say that $\nu$ is \textbf{$\mu$-stationary} (with respect to the action) if
	\begin{equation}
		\label{intro: stationary measure}
		\nu = \int_\gamma  \gamma_* \nu d \mu(\gamma).
	\end{equation}
\end{definition}

It is easy to see that any invariant measure is stationary with respect to any probability measure on $\Gamma$, but the inverse is, of course, not true. Being a $\mu$-stationary measure is, evidently, a much weaker condition. Stationary measures exist in very general settings, unlike invariant ones, but they are no less important, as they are closely related to the Poisson boundaries and long-term dynamics of random walks on groups. 

Given an admissible random walk $(X_n)$ on a non-elementary discrete subgroup $\Gamma \subset PSU(1,1)$ with a finite first moment, we know (due to Furstenberg (\cite{furstenberg71}) and Kaimanovich (\cite{kaimanovich2000poisson})) that $(X_n)$ converges to the Poisson boundary almost surely. The respective pushforward of the resulting measure to $S^1$ via the identification $\partial \Gamma \simeq S^1$ with the Gromov boundary yields a (unique) $\mu$-stationary measure $\nu_\mu$ with respect to the action of $\Gamma$, called the \textbf{hitting measure} of the random walk. A big open problem in measured group theory is to understand when hitting measures are singular or absolutely continuous with respect to the Lebesgue measure on $S^1$. In particular, recall the Kaimanovich-le Prince's singularity conjecture:

\begin{conjecture}[\cite{kaimanovich2011matrix}]
	\label{Fuchsian singularity conjecture}
	For every finite-range admissible random walk $(X_n)$ generated by a probability measure $\mu$ on a cocompact Fuchsian group $\Gamma$, the hitting measure $\nu_\mu$ is singular with respect to the Lebesgue measure on $S^1 \simeq \partial \Gamma$.
\end{conjecture}

This conjecture is known to hold for non-cocompact lattices due to \cite{guivarch1990}, and the author's thesis \cite{mythesis} provides affirmative results for nearest-neighbour random walks on cocompact Fuchsian groups, but the conjecture is still widely open, as we did show that even the recently developed geometric ideas are insufficient to completely settle the conjecture.

In our paper we will study the actions of subgroups $\Gamma \subset G = PSU(1,1)$ induced by the action of $G$ on $S^1$ via hyperbolic isometries, or, more concretely, M\"{o}bius transformations. We aim to present a complex-analytic framework which, as we believe, can unify the majority existing results about stationary measures on $S^1$ with respect to the action of $\Gamma$ and probability measures satisfying a finite first moment condition. Keep in mind that the analysis will be different depending on several factors:

\begin{itemize}
	\item Whether $\mu$ has finite support or not,
	\item If $\mu$ is infinitely supported, the moment conditions on $\mu$ will matter (first moment, exponential moment, superexponential moment, and so on...),
	\item Whether the subgroup of $G$ generated by the support of $\mu$ is discrete or not,
	\item If the generated subgroup is discrete, whether it is of first or second type,
	\item And, finally, if it is of first kind, whether it is cocompact or not.
\end{itemize}

First results about stationary measures and Poisson boundaries for discrete subgroups of $SL_n(\mathbb{R})$ were established by Furstenberg in \cite{furstenberg1963noncommuting}. In particular, the question of when the Lebesgue measure is $\mu$-stationary was first studied by Furstenberg as well in \cite{furstenberg71}. Pure Fourier-like approaches were independently demonstrated by \cite{Bourgain2012} and \cite{MR2969625}, which allow us to study $\mu$-stationary measures for \textbf{dense} subgroups of $PSU(1,1)$. However, their methods do not apply for discrete groups and are quite delicate with respect to the initial data, requiring complicated number-theoretic and analytic methods to properly apply. There have been multiple improvements to Bourgain's approach, see \cite{lequen2022absolutely} and \cite{kittle2023absolutely} for latest examples, but they still do not apply to the discrete case and non-finite supports. Finally, we want to mention recent attempts to understand the structure of harmonic and Patterson-Sullivan measures using thermodynamic formalism, for example, \cite{garcía2023dimension} and \cite{cantrell2022invariant}. Once again, these approaches are not universal, as Garc\`{i}a-Lessa's paper does not generalize to first-kind Fuchsian groups, and the thermodynamic approach of Cantrell-Tanaka provides considerably more information for Patterson-Sullivan measures than harmonic measures.

As one can see, up until now there was no single method which unified all above settings, and until very recently, the general consensus was that no such method should have exist, in light of the incredible variety of techniques used to study different settings. 

\subsection{Main results}
Inspired by the standard techniques used to study affine self-similar measures on $\mathbb{R}^n$, and their respective Parseval frames, papers of R.S.Strichartz (see \cite{strichartzI} and sequels), together with \cite{denseanalytic} and more recent papers of E.Weber and J.Herr \cite{weber2017paleywiener} and \cite{axioms6020007}, we have developed a promising approach which, in theory, could unify many standard results about stationary measures on $S^1$ with respect to the action of $PSU(1,1)$. The idea is to consider an appropriate integral transform on $S^1$ which respects the action of $PSU(1,1)$ and ``preserves'' \eqref{intro: stationary measure}. The Fourier transform is known to not respect this action, and the resulting exponential terms $e^{(az +b) / (cz+d)}$ are difficult to control. The Helgason-Fourier transform seems to be a better candidate, but integrating the powers of the Poisson kernel $(z, \xi) \mapsto \left( \frac{1 - |z|^2}{|z - \xi|^2}\right)^{\lambda i + 1}$ against a stationary measure does not actually preserve \eqref{intro: stationary measure} in a way we want. Essentially, given a $\mu$-stationary measure $\nu$ on $S^1$, one can easily check that the resulting smooth eigenvector of the hyperbolic Laplacian
\[
\psi(z) := \frac{1}{2\pi} \int_{0}^{2\pi} \left( \frac{1 - |z|^2}{|z - e^{i t}|^2}\right)^{\lambda i + 1} d\nu(t)
\]
does not exhibit any nice properties with respect to the action of $PSU(1,1)$, unlike what we see for Patterson-Sullivan measures. However, replacing the Poisson kernel with its logarithm, which is closely related to the Busemann cocycle, does the trick, turning a multiplicative relation into an additive one. The resulting functional equation \eqref{intro: functional equation} serves as a proper holomorphic version of \eqref{intro: stationary measure}, and, in a way, it allows us to change the perspective, as we shift from the measurable setting to a holomorphic functions on $\mathbb{D}$, granting access to the vast complex-analytic machinery.

Before stating our results, we would like to point out that it is sufficient to study pure $\mu$-stationary measures due to the fact that the action of $PSU(1,1)$ respects the Lebesgue decomposition, this is a standard reduction.

Our main result is a following necessary condition for stationarity.

\begin{theorem}
	\label{T:main result}
	Let $\mu$ be a complex finite Borel measure on $G=PSU(1,1)$ satisfying the following moment condition:
	\begin{equation}
		\int_{G} \log\left( \frac{1 + |\gamma.z| }{1 - |\gamma.z|} \right)  d |\mu| (\gamma) < \infty
	\end{equation}
	for any $z \in \mathbb{D}$.	Then a probability measure $\nu$ on $S^1$ is $\mu$-stationary implies that the Cauchy transform 
	\[
	f_\nu(z) := \frac{1}{2\pi} \int_{0}^{2\pi} \dfrac{d\nu(t)}{e^{it} - z}
	\] 
	satisfies
	\begin{equation}
		\label{intro: functional equation}
		\int_G f_\nu(\gamma^{-1}.z)(\gamma^{-1})'(z)  d\mu(\gamma) - f_\nu(z) = \int_G \frac{d \mu(\gamma)}{z - \gamma.\infty}.
	\end{equation}
	for every $z \in \mathbb{D}$. Moreover, if $\mu$ is countably supported then the above holds for all \\ $z \in \overline{\mathbb{C}} \setminus \{ \mathbb{T} \cup \{ \gamma.\infty \}_{\gamma \in \text{supp} \mu } \}$.
\end{theorem}

\textbf{Remark.} We don't claim that this is a sufficient condition for any $\mu$, as we, predictably, expect to lose some information by ``cutting'' $\nu$ in half and applying $\mu$-stationarity. Moreover, as the reader will see, this direction turns out to be more than enough for our purposes.

The power of this theorem lies in the fact that we managed to successfully transform a measurable functional equation on the circle into a holomorphic condition on the unit disk, which allows us to make use of powerful complex-analytic techniques.

We are able to extract the most amount of information from \eqref{intro: functional equation} for \textbf{countably supported probability} measures $\mu$. Before stating the corollary, let us recall the \textbf{Blaschke condition} for a sequence $\{ z_n \} \subset \mathbb{D}$:

\begin{equation}
	\label{Blachke condition}
	\sum_{\gamma \in \text{supp} \, \mu} (1 - |\gamma.0|) < \infty.
\end{equation}

We will say that $\mu$ satisfies the Blaschke condition if and only if $\{ \gamma.0 \}_{\gamma \in \text{supp} \, \mu}$ satisfies \eqref{Blachke condition}.

\begin{corollary}
	\label{C:main corollary}
	Let $\mu$ be a countably supported probability measure on $PSU(1,1)$ with a finite first moment.
	\begin{enumerate}
		\item Assume that $\mu$ satisfies the Blaschke condition, and there is an element $\gamma \in \text{supp} \, \mu$ with $\gamma.0 \ne 0$. Then there are no entire solutions to \eqref{intro: functional equation}. In particular, there are no $\mu$-stationary measures with the Fourier series $\nu \sim \sum_{k \in \mathbb{Z}} a_k e^{i k t}$ with $\limsup_{n \rightarrow \infty} |a_k|^{1/k} = 0.$
		\item Assume that $\limsup\limits_{n \rightarrow \infty} \left\| \int_G \dfrac{d \mu^{*n}(\gamma)}{z - \gamma.\infty} \right\|_1 = \infty$, where $||\cdot||_1$ stands for the norm in $H^1(\mathbb{D})$. Then there are no $\mu$-stationary measures with $L^{1+\varepsilon}(S^1, m)$-density for any $\varepsilon > 0$.
	\end{enumerate}
\end{corollary}

Equation \eqref{intro: functional equation} gives us quite a lot of insight into the measures $\mu$ for which the (normalized) Lebesgue measure is $\mu$-stationary.

\begin{corollary}
	\label{intro: Lebesgue is stationary}
	Let $\mu$ be a finite Borel measure supported on a discrete subgroup $\Gamma \subset PSU(1,1)$ with a finite first moment. Let's call a measure $\mu$ on $PSU(1,1)$  the \textbf{Furstenberg measure} if the normalized Lebesgue measure on $S^1$ is $\mu$-stationary.
	\begin{enumerate}
		\item If $\mu$ is a Furstenberg measure, then
		\begin{equation}
			\int_G \frac{d \mu(\gamma)}{z - \gamma.\infty} = 0, \quad |z| < 1.
		\end{equation}
		\item If $\mu$ is a Furstenberg measure, then
		\[
		\limsup_{n \rightarrow \infty} |\mu(\gamma_n)|^{1/n} = 1.
		\]
		\item (Brown-Shields-Zeller) Suppose $\mu$ is a Furstenberg measure. Then $\{\gamma.0\}_{\gamma \in \text{supp} \, \mu}$ is non-tangentially dense in $\mathbb{T}$, which means that $m$-\textbf{almost every} point $\xi \in \mathbb{T}$ can be approached by a subsequence $\gamma_n.0$ inside a Stolz angle $\{ z \in \mathbb{D} : \frac{|z - \xi|}{1 - |z|} < \alpha \}$ for some $\alpha > 1$. As a corollary from \cite[Remark 2]{brownsums}, we get
		\[
		\sum_{\gamma \in \text{supp}\, \mu} (1 - |\gamma.0|) = \infty.
		\]
	\end{enumerate}
\end{corollary}

\textbf{Remark.} One fascinating detail of this theorem lies in the fact that the Brown-Shields-Zeller theorem detects the divergence of the Poincar\'e series for first-kind groups at the critical exponent $\delta = 1$, as
\[
\sum_{\gamma} e^{- d_{\mathbb{H}^2}(0, \gamma.0)} = \sum_{\gamma} e^{-\ln\left( \frac{1 + |\gamma.0|}{1 - |\gamma.0|} \right)} \sim \sum_{\gamma} (1 - |\gamma.0|).
\]

Finally, as a corollary from Fatou's theorem, we get a functional-analytic necessary condition for existence of $\mu$-stationary measures with $L^p$-density for $1 < p < \infty$.

\begin{corollary}
	\label{functional-analytic necessary condition}
	Let the support of $\mu$ satisfy the Blaschke condition. Then for any $\mu$-stationary measure $\mu$ with $L^p$-density for $1 < p < \infty$, we have
	\[
	f_\nu(z) \in (\overline{T_\mu^*(B_\mu H^q)})^{\perp} \subset H^p,
	\]
	where
	\begin{itemize}
		\item $T_\mu(f) := \sum_{\gamma} \mu(\gamma) (f \circ \gamma^{-1}) (\gamma^{-1})' - f$ is considered as a bounded linear operator \\ $T_\mu : H^p(\mathbb{D}) \rightarrow H^p(\mathbb{D})$, and $\frac{1}{p} + \frac{1}{q} = 1$,
		\item $B_\mu(z)$ is the Blaschke function corresponding to the support of $\mu$:
	\end{itemize}
	\[
	B_\mu(z) := \prod_{\gamma \in \text{supp}(\mu)} \gamma^{-1}(z).
	\]
	In particular, if $T_\mu^*(B_\mu H^q)$ is dense in $H^q$, then there are no $\mu$-stationary measures with $L^p$-density.
\end{corollary}

Finally, we make use of A. Aleksandrov's approximation results in $H^p(\mathbb{D})$ for $p < 1$ to get the following result.

\begin{theorem}
	\label{intro: singularity conjecture}
	Let $(X_n)$ be an admissible random walk on a non-elementary discrete subgroup \\ $\Gamma \leq PSU(1,1)$ with the finite first moment generated by the probability measure $\mu$. If $\nu$ is the $\mu$-stationary measure, then $f_\nu(z) \in H^p \cap \overline{H^p_0}$ for all $0 < p < 1$. In particular, there are only three possible outcomes:
	\begin{enumerate}[label=(\arabic*)]
		\item $\nu$ is the Lebesgue measure,
		\item $\nu$ admits an $L^1(S^1)$-density which does not belong to $L^{(1+\varepsilon)}(S^1)$ for any $\varepsilon > 0$,
		\item $\nu$ is singular with respect to the Lebesgue measure.
	\end{enumerate}
	If $\mu$ satisfies the Blaschke condition, then (1) cannot happen, that is, $\mu$ is not a Furstenberg measure. If, in addition, $\mu$ has finite superexponential moment, then the hitting measure is singular with respect to the Lebesgue measure.
\end{theorem}


Corollary \ref{C:main corollary}.1 strictly strengthens the very last remark in \cite{Bourgain2012}, where it was proven that the Lebesgue measure is never stationary with respect to finitely supported measures on $PSU(1,1)$. Corollary \ref{C:main corollary}.2, in theory, provides a purely computational heuristic to showing singularty of stationary measures, as for lattices in $PSU(1,1)$, one expects the poles to converge to $\mathbb{T}$, whereas for dense subgroups one would expect the poles to accumulate inside $\mathbb{D}$, thus forcing the $H^1$-norms to stay bounded. 


Corollary \ref{intro: Lebesgue is stationary} provides several new insights into Furstenberg measures on $PSU(1,1)$. In particular, as our approach deals with signed and complex measures, we are able to talk about complex Furstenberg measures, which is not possible with any geometric approaches. In particular, we obtain Borel sums with poles in the orbits of a non-cocompact lattice which vanishes in $\mathbb{D}$, despite the fact that such counterexamples should be impossible due to Guivarch'-le Jan (\cite{guivarch1990}). The catch is that our condition is only a necessary one; and the Brown-Shields-Zeller theorem does not control the moments of the resulting coefficients. We also exhibit the first known result restricting the moment conditions of a Furstenberg measure, once again, improving on \cite{Bourgain2012}. The notion of a non-tangential limit seems to be key in this approach. Finally, we remark that studying positive Furstenberg measures should be possible using techniques in \cite{bonsall1989vanishing} and \cite{hayman1990bases}, as they deal with Borel-like series having strictly positive coefficients.

Corollary \ref{functional-analytic necessary condition} provides a pretty significant restriction on the stationary measures in the $L^p$-class for $1 < p < \infty$, and, in theory, the images with respect to the adjoint operator $T_\mu^*$ can be computed explicitly for any measure $\mu$ satisfying the Blaschke condition.

Finally, Theorem \ref{intro: singularity conjecture} affirmatively resolves the singularity conjecture by exploiting the fact that the random walk induces an $H^p$-approximation of the Cauchy transforms of $\mu^{*n}$ to the Cauchy transforms of the hitting measure $\nu$. While the Cauchy transforms of $\mu^{*n}$ do not(!) belong to a special closed subspace $H^p \cap \overline{H^p_0} \subset H^p$ for all $p < 1$ they approach it in the $H^p$-metric, so the limit has to belong to $H^p \cap \overline{H^p_0}$ as well. This subspace does not contain Cauchy-Szeg\H{o} transforms of non-Lebesgue absolutely continuous measures with ``nice'' densities, and we can use the moment condition to invoke \cite{blachere2011harmonic} and eliminate this case. The proof of the main statement is mostly independent of Theorem \ref{T:main result}, we only need the functional equation to determine whether the Lebesgue measure can be stationary.

The structure of the paper is as follows. In Section \ref{Preliminaries} we recall all necessary facts about transformations $PSU(1,1)$ and provide a brief recap of complex-analytic tools we are going to use. In Section \ref{The log-Poisson transform of a stationary measure} we introduce an appropriate integral transform which fully respects the action of $PSU(1,1)$ to obtain a holomorphic necessary condition for $\mu$-stationarity, thus proving Theorem \ref{T:main result}. In Section \ref{Squeezing water from a stone} we extract the most we can from the resulting equation, using state-of-the-art techniques related to generalized analytic continuations. Section \ref{The proof of the singularity conjecture} is dedicated to proving Theorem \ref{intro: singularity conjecture} by establishing several results about approximation by rational functions/Borel series in $H^p$ for $0 < p < 1$.

%\textbf{Remark.} This is very much work in progress -- we believe that the established connection between the structure of stationary measures and closed invariant subspaces with respect to the backward shift goes much, much deeper. We will outline some of the open questions in the concluding section of this preprint.

\subsection{Acknowledgements}
I would like to thank Ilyas Bayramov, Ivan Nikitin, Ilia Binder, Kunal Chawla, Mathav Murugan, Pablo Shmerkin, Lior Silberman and Omer Angel for useful discussions. Also I am tremendously grateful to Alexander Kalmynin, Giulio Tiozzo and Tianyi Zheng for reading the preliminary version of this preprint and for providing continued support and encouragement throughout the past year. Finally, I would like to thank Giulio Tiozzo for organizing a short visit to the Fields institute on Feb 19--22, 2024. 

\section{Preliminaries}
\label{Preliminaries}
\subsection{Everything you need to know about isometries of the disk model of $\mathbb{H}^2$}
In this subsection we will recall basic facts about $PSU(1,1)$ considered as a isometry group of the disk model $\mathbb{D} = \{ |z| < 1 \}$ of the hyperbolic plane. 
\begin{definition}
	\[
	PSU(1,1) = \left\lbrace z \mapsto \frac{az + b}{\overline{b} z + \overline{a}} : a, b \in \mathbb{C}, |a|^2 - |b|^2 = 1 \right\rbrace.
	\]
\end{definition}

From the definition it is evident that every transformation in $PSU(1,1)$ can be represented by a matrix $\begin{pmatrix}
	a & b \\ \overline{b} & \overline{a}
\end{pmatrix}$ (mod scalar matrices). In particular, if $\gamma(z) = \frac{az + b}{\overline{b} z + \overline{a}}$ then $\gamma^{-1}(z) = \frac{\overline{a} z - b}{ -\overline{b}z + a}$.

Also, it will turn out that sometimes working with $\infty$ as a basepoint is more conventient than choosing $0 \in \mathbb{H}^2$, we will use

\begin{equation}
	\label{from inside to outside}
	\overline{\gamma\left( \overline{z}^{-1} \right)} = \frac{1}{\gamma(z)}, \quad z \in \overline{\mathbb{C}},
\end{equation}

and, as a simple corollary,

\begin{equation}
	\label{zero-infty}
	\gamma.\infty = \frac{a}{\overline{b}} = \left( \frac{\overline{b}}{a}\right)^{-1} = (\overline{\gamma.0})^{-1}.
\end{equation}

\begin{lemma}
	Let $\gamma(z) = \frac{az + b}{\overline{b} z + \overline{a}}$. then
	\[
	\gamma'(z) = \frac{1}{(\overline{b} z + \overline{a})^2}.
	\]
\end{lemma}

\begin{proof}
	\[
	\gamma'(z) = \frac{a ( \overline{b} z + \overline{a} ) - (az + b)\overline{b}}{(\overline{b} z + \overline{a})^2} = \frac{|a|^2 - |b|^2}{(\overline{b} z + \overline{a})^2} = \frac{1}{(\overline{b} z + \overline{a})^2}.
	\]
\end{proof}


\begin{definition}
	\begin{equation}
		\label{pseudo-Busemann cocycle}
		\hat{c}_B(z, \xi) := \log\left( \frac{1- |z|^2}{|z - \xi|^2} \right), \quad |z| < 1, |\xi| = 1.  
	\end{equation}
\end{definition}


\textbf{Remark.} This notation is motivated by the explicit form of the \textbf{Busemann cocycle} for discrete subgroups of $PSU(1,1)$ acting geometrically on $\mathbb{H}^2$.

Let us prove some properties of $\hat{c}_B(z, \xi)$ first.

\begin{lemma}
	For every $\gamma \in PSU(1,1)$ and $|z| < 1$ we have
	
	\begin{equation}
		\label{eq1}
		\log(1 - |\gamma.z|^2) - \log(1 - |z|^2) = \log(|\gamma'(z)|), \quad z \in \overline{\mathbb{C}},
	\end{equation}
 	and
	\begin{equation}
		\label{cocycle}
		\hat{c}_B(\gamma.z, \gamma.\xi) = \hat{c}_B(z, \xi) - \hat{c}_B(\gamma^{-1}.0, \xi), \quad |z| < 1, |\xi| = 1.
	\end{equation}
\end{lemma}

\begin{proof}
Let $\gamma(z) := \frac{a z + b}{\overline{b} z + \overline{a}}$. Then
\[
\begin{aligned}
	\dfrac{1 - |\gamma.z|^2}{1 - |z|^2} &= \dfrac{1 - \left| \frac{az + b}{\overline{b} z + \overline{a}} \right|^2 }{1 - |z|^2} = \dfrac{|\overline{b} z + \overline{a}|^2 - \left| az + b \right|^2 }{(1 - |z|^2) |\overline{b} z + \overline{a}|^2} = \\
	&= \dfrac{(\overline{b} z + \overline{a})(b \overline{z} + a) - (az + b)(\overline{a} \overline{z} + \overline{b}) }{(1 - |z|^2) |\overline{b} z + \overline{a}|^2} = \\ 
	&= \frac{1 - |z|^2}{(1 - |z|^2) |\overline{b} z + \overline{a}|^2} = \frac{1}{|\overline{b} z + \overline{a}|^2} = |\gamma'(z)|.
\end{aligned}
\]
This implies \eqref{eq1}. As for \eqref{cocycle}, observe that for every $\gamma(z) = \frac{az + b}{\overline{b} z + \overline{a}}$ we have
\[
\begin{aligned}
	|\gamma.z - \gamma.\xi|^2 &=  \left( \frac{az + b}{\overline{b} z + \overline{a}} - \frac{a \xi + b}{\overline{b} \xi + \overline{a}}\right) \left( \frac{\overline{a}\overline{z} + \overline{b}}{b \overline{z} + a} - \frac{\overline{a}\overline{\xi} + \overline{b}}{b \overline{\xi} + a} \right) = \\ 
	&= |\gamma'(z)| |\gamma'(\xi)| |z - \xi|^2,
\end{aligned}
\]
and
\[
\hat{c}_B(\gamma^{-1}.0, \xi) = \log \left( \frac{1 - |\frac{b}{a}|^2}{|-\frac{b}{a} - \xi|^2} \right) = \log \left( \frac{|a|^2 - |b|^2}{|-b - a\xi|^2} \right) = \log(|\gamma'(\xi)|).
\]
Therefore,
\[
\begin{aligned}
	\hat{c}_B(\gamma.z, \gamma.\xi) &= \log\left( \frac{1- |\gamma.z|^2}{|\gamma.z - \gamma.\xi|^2} \right) = \log\left( \frac{(1- |z|^2) |\gamma'(z)|}{|\gamma'(z)| |\gamma'(\xi)| |z - \xi|^2} \right) = \\ &= \log \left( \frac{1 - |z|^2}{|\gamma'(\xi)| |z - \xi|^2} \right) = \hat{c}_B(z, \xi) - \log(|\gamma'(\xi)|) = \\ &= \hat{c}_B(z, \xi) - \hat{c}_B(\gamma^{-1}.0, \xi)
\end{aligned}
\]
\end{proof}

\textbf{Remark.} Compare with the property of being a $2$-cocycle:
\[
c(gh, x) = c(g, hx) + c(h, x).
\]

Finally, recall that for any $\gamma(z) = \frac{a z + b}{\overline{b} z + \overline{a}}$ with $|a|^2 - |b|^2 = 1$ we have
\begin{equation}
	\label{poles of the log derivative}
		\frac{1}{2} \frac{\gamma''(z)}{\gamma'(z)} = \frac{1}{2} \frac{-2\overline{b}}{(\overline{b}z + \overline{a})^3} \left( \frac{1}{(\overline{b}z + \overline{a})^2}\right)^{-1} = \frac{- \overline{b}}{\overline{b}z + \overline{a}} = - \frac{1}{z + \frac{\overline{a}}{\overline{b}}} = - \frac{1}{z - \gamma^{-1}.\infty}.
\end{equation}

\subsection{Complex-analytic prerequisites}
We will heavily rely on standard complex-analytic techniques related to Cauchy transforms and generalized analytic continuation, we refer to standard textbooks on these topics: \cite{Shapiro1968}, \cite{cimahardy}, \cite{book:738388}.

Let us denote $\mathbb{D} = \{ |z| < 1 \}$ and $\mathbb{D}_e := \overline{\mathbb{C}} \setminus \overline{\mathbb{D}}$. Given a domain $U \subset \overline{\mathbb{C}}$, we will denote the space of holomorphic functions on $U$ by $\mathfrak{H}(U)$ and the space of meromorphic functions on $U$ by $\mathfrak{M}(U)$.

\begin{definition}
	Let $0 < p < \infty$. The \textbf{Hardy space} $(H^p(\mathbb{D}), || \cdot ||_p)$ is a space of holomorphic functions on $\mathbb{D}$ defined as follows.
	\[
	H^p(\mathbb{D}) = \left\lbrace  f \in \mathfrak{H}(\mathbb{D}) \ | \ ||f||_p := \sup_{0 < r < 1} \left( \frac{1}{2 \pi} \int_{0}^{2 \pi} |f(r e^{it})|^p dt \right)^{1/p} < \infty \right\rbrace.
	\] 
	If $p = \infty$, then we define $(H^\infty(\mathbb{D}), ||f||_\infty)$ as the space of bounded holomorphic functions on $\mathbb{D}$ equipped with the sup-norm.
	
	Finally, we define $H^p(\mathbb{D}_e) := \{ z \mapsto f(1/z) : f \in H^p(\mathbb{D})\}$, with $H^p_0(\mathbb{D}) \subset H^p(\mathbb{D}_e)$ denoting functions vanishing at infinity.
\end{definition} 

It is well-known that for $1 \le p \le \infty$ the function $||\cdot||_p : H^p(\mathbb{D}) \rightarrow \mathbb{R}_{\ge 0}$ defines a norm, so the respective Hardy spaces $H^p(\mathbb{D})$ are Banach spaces for $1 \le p \le \infty$. For $0 < p < 1$ the Hardy spaces $H^p(\mathbb{D})$ admit a complete translation-invariant metric defined by $d(f, g) := ||f - g||^p_p$, but the topology it defines is not non-locally convex.

\begin{definition}
	\label{D:non-tangential convergence}
	A sequence of points $\{z_n\} \subset \mathbb{D}$ is said to \textbf{non-tangentially} converge to $\xi \in \partial \mathbb{D}$ if there exists a \textbf{Stolz angle} $\{ \frac{|\xi - z|}{1 - |z|} \le M \}$ and $N > 0$ such that $z_n \rightarrow \xi$ and $z_n \subset A$ for $n > N$.
\end{definition}

We will frequently use the following classical theorems.

\begin{theorem}[Fatou's theorem]
	\label{T: Fatou theorem}
	Every holomorphic function $f \in H^p(\mathbb{D})$ for $0 < p \le \infty$ admits a non-tangential limit $f(\zeta)$ for $Leb$-almost every $\zeta \in S^1$ which belongs to $L^p(S^1, Leb)$. Moreover, for $0 < p < \infty$ we have
	\[
	||f||_p = \left( \frac{1}{2 \pi} \int_{0}^{2 \pi} |f(e^{i t})|^p dt\right)^{1/p} .
	\]
\end{theorem}

\begin{theorem}[F. Riesz - M. Riesz]
	\label{T: Riesz-Riesz theorem}
		For $p \ge 1$ we have a complete realization of the Hardy spaces $H^p(\mathbb{D})$ as subspaces $L^p(S^1)$: these are exactly the functions with vanishing negative Fourier coefficients.
\end{theorem}

Let us briefly list some examples of holomorphic functions in Hardy spaces.

\begin{example}
	\indent
	\begin{itemize}
		\item Analytic polynomials $p(z) = a_0 + \dots + a_n z^n$ are dense in $H^p(\mathbb{D})$ for all $0 < p < \infty$, and are $wk^*$-dense in $H^\infty(\mathbb{D})$. (\cite[Theorem 1.9.4]{book:738388})
		\item If $0 < p < q \le \infty$, then $H^q(\mathbb{D}) \subsetneq H^p(\mathbb{D})$.
		\item For any $z_0 \in \mathbb{D}_e$ and $k > 0$, we have $\frac{1}{(z - z_0)^k} \in H^\infty(\mathbb{D})$, hence $\frac{1}{(z - z_0)^k} \in H^p(\mathbb{D})$ for any $0 < p \le \infty$. Keep in mind that this will not hold for any $|z_0| = 1$, see later examples.
	\end{itemize}
\end{example}

%\begin{definition}
%	Let $f$ be a meromorphic function on $\mathbb{D}$ ($\mathbb{D}_e$, resp.). If the limit $\lim\limits_{r \rightarrow 1^-} f(r e^{it})$ ($\lim\limits_{r \rightarrow 1^+} f(r e^{it})$ resp.) exists Leb-almost everywhere on $\mathbb{T}$, then we say that $f$ admits a \textbf{non-tangential limit} on the boundary.
%\end{definition}
%
%\begin{definition}
%	Let $f$ be a meromorphic function on $\mathbb{D}$. If there exists a function $T_f$ which is meromorphic on $\mathbb{D}_e$ such that the non-tangential limits of $f$ and $\tilde{f}$ coincide Leb-almost everywhere, then we say that $f$ is \textbf{pseudocontinuable}, and $T_f$ is a \textbf{pseudocontinuation} of $f$, and vice versa.
%\end{definition}
%
%In our paper we will use several important results about non-tangential limits and pseudocontinuations.
%
%\begin{theorem}[Lusin-Privalov, \cite{privalov1956randeigenschaften}]
%	If $f$ is pseudocontinuable, then its pseudocontinuation is unique.
%\end{theorem}
%
%As a corollary, we get that pseudocontinuations are compatible with analytic continuations. The next theorem is a well-known fact about non-tangential limits of functions in the Hardy spaces $H^p(\mathbb{D})$ and $H^p(\mathbb{D}_e)$.

\begin{definition}
	Let $\nu$ be a complex finite Borel measure on $S^1$. Then its \textbf{Cauchy transform} is the integral
	\begin{equation}
		C_\nu(z) := \frac{1}{2\pi} \int_0^{2 \pi} \dfrac{d \nu(t)}{1 - e^{-it} z}.
	\end{equation}
	Its \textbf{Cauchy-Szeg\H{o} transform} is the integral
	\begin{equation}
		f_\nu(z) := \frac{1}{2\pi} \int_0^{2 \pi} \dfrac{d \nu(t)}{e^{it} - z}.
	\end{equation}
\end{definition}

The properties of $C_\nu(z)$ as a holomorphic function on $\mathbb{D}$ strongly depend on $\nu$ itself, but the following theorem of Smirnov ensures that we at least end up in $H^p(\overline{C}  \setminus \mathbb{T})$ for $p < 1$.

\begin{theorem}[Smirnov]
	\label{T:Smirnov}
	Let $f(z) = C_\nu(z)$ for some complex finite Borel measure $\nu$. Then $f \in H^p(\mathbb{D})$ and $f \in H^p(\mathbb{D}_e)$ for all $0 < p < 1$.
\end{theorem}

We can do better if we know that $\nu \ll Leb$ due to a theorem of M. Riesz.

\begin{theorem}[Riesz]
	\label{Cauchy of Lp}
	Let $\nu$ be an absolutely continuous measure on $S^1$ with the $L^p$-density for $1 < p < \infty$. Then $C_\nu(z) \in H^p(\mathbb{D})$.
\end{theorem}

\textbf{Remark.} This theorem cannot not hold for $p = 1, \infty$, as it is well-known that there are no continuous projections $L^1 \rightarrow H^1$ and $L^\infty \rightarrow H^{\infty}.$

\begin{example}
	\indent
	\begin{itemize}
		\item 	Consider $f(z) = (1 - z)^{-1}$. This is the Cauchy transform of the Dirac delta measure $\delta_1$, so $f(z) \in H^p(\mathbb{D})$ for all $0 < p < 1$, but $f(z) \notin H^{1}(\mathbb{D})$. Both statements are slightly tricky to prove directly, but a vital idea is that the integral
		\[
		\int_{0}^1 \frac{dx}{x^p}
		\]
		converges for $p < 1$ and diverges for $p = 1$.
		\item This suggests that for $0 < p < 1$ it makes sense to talk about the closure of all simple poles $z \rightarrow \frac{1}{1 - e^{it}z}$ on $\mathbb{T}$, and it turns out that this closure admits a very nice description:
		\begin{equation}
			\text{span}^{H^p} \left\lbrace \frac{1}{1 - e^{it}z} \right\rbrace := H^p \cap \overline{H^p_0},
		\end{equation}
		where by $H^p \cap \overline{H^p_0}$ we denote the subspace of all functions $f(z) \in \mathcal{H}(\overline{C} \setminus \mathbb{T})$, such that both inner and outer components lie in $H^p$, $f(\infty) = 0$ and inner + outer non-tangential limits a.e. exist and coincide. 
		
		
	\end{itemize}
\end{example}
\textbf{Remark.} This exotic subspace has trivial intersections with $H^{1 + \varepsilon}(\mathbb{D})$ for all $\varepsilon > 0$, and it is of critical importance when studying harmonic measures for finitely supported random walks on discrete subgroups. Our strategy exploits the finiteness of the support to exhibit a rational approximation (with simple poles on $\mathbb{T}$) of the Cauchy-Szeg\H{o} transform of the hitting measure, which will force the transform to be in this subspace.

We will also need some facts about weighted composition operators with respect to M\"obius transformations.

\begin{theorem}
	\label{weighted composition ops}
	Let $\gamma \in PSU(1,1)$ and consider the operator
	\[
	V_\gamma(f)(z) := f(\gamma^{-1}(z)) (\gamma^{-1})'(z).
	\] 
	\begin{itemize}
		\item For every $w \in \mathbb{C}$ we have
		\[
		V_\gamma \left( \frac{1}{w - z} \right) = \frac{1}{\gamma.w - z} - \frac{1}{\gamma.\infty - z}.
		\]
		\item $V_\gamma : H^p(\mathbb{D}) \rightarrow H^p(\mathbb{D})$ is a bounded linear operator for $1 \le p \le \infty$, being an isometry for $p = 1$.
	\end{itemize}
\end{theorem}

The first identity quickly follows from a residue computation via (complex) change of variables. The proof of the second statement is a classical application of the Littlewood subordination theorem (see \cite{book:2131218} for details). 

\section{The log-Poisson transform of a stationary measure}
\label{The log-Poisson transform of a stationary measure}
\begin{definition}
	Let $\mu$ be a complex finite Borel measure on $G = PSU(1,1)$. Then a probability measure $\nu$ on $S^1$ is $\mu$-stationary if
	\begin{equation}
		\label{stationary measure}
		\int_G \gamma_*(\nu) d\mu(\gamma) = \nu.
	\end{equation}
\end{definition}

\textbf{Remark.} We put minimal restrictions on $\mu$, for example, we allow measures $\mu$ which are absolutely continuous with respect to the Haar measure on $G$.

Let $\nu$ be a Borel probability measure on $S^1$. Consider the function (see \eqref{pseudo-Busemann cocycle} for notation)
\begin{equation}
	\label{def of the special function}
	p_\nu(z) := \int_{S^1} \hat{c}_B(z, \xi) d \nu(\xi).
\end{equation}

It is easily seen that for any $|z| < 1$ this is a well-defined function as $\hat{c}_B(z, \xi)$ is bounded with respect to $\xi$ on $S^1$. However, it is not always non-negative, as the Poisson kernel itself might take values less than $1$.

\begin{lemma}
	Let $\mu$ be a complex finite Borel measure satisfying the \textbf{finite first moment} condition:
	\begin{equation}
		\label{E:finite first moment}
		\int_{G} \log\left( \dfrac{1 + |\gamma.z|}{1 - |\gamma.z|} \right) d |\mu|(\gamma)  < \infty
	\end{equation} 
	for any $|z| < 1$. If $\nu$ is a $\mu$-stationary probability measure, then
	\begin{equation}
		\label{Poisson stat}
		\int_{G} p_\nu(\gamma^{-1}.z) d \mu(\gamma) - p_\nu(z) = const
	\end{equation}
	for all $|z| < 1$.
\end{lemma}

\begin{proof}
	First of all, we need to resolve any convergence issues. Finite first moment ensures that the cocycle is absolutely integrable for all $|z| < 1$:
	\begin{equation}
		\label{first moment ensures DCT}
		\begin{aligned}
			\int_G \int_{S^1} |\hat{c}_B(\gamma^{-1}.z, \xi)| d\nu(\xi) d |\mu|(\gamma)  &= \int_G \int_{S^1} \left|  \log \left( \dfrac{1 - |\gamma^{-1}.z|^2}{|\gamma^{-1}.z - \xi|^2} \right) \right| d\nu(\xi)  d |\mu|(\gamma)  \le \\ 
			&\le \int_G   \log \left( \dfrac{1 - |\gamma.z|^2}{(1-|\gamma.z|)^2} \right) d|\mu|(\gamma) = \\ 
			&= \int_G  \log \left( \dfrac{1 + |\gamma.z|}{1-|\gamma.z|} \right) d|\mu|(\gamma) < \infty.
		\end{aligned}	
	\end{equation}
	Therefore, we are able to make use of the Fubini's theorem and DCT. If $\nu$ were to be $\mu$-stationary, then
	\[
	\begin{aligned}
		p_\nu(z) \stackrel{\eqref{stationary measure}}{=} & \int_{G} \left( \int_{S^1} \hat{c}_B(z, \gamma.\xi) d \nu(\xi) \right)  d \mu(\gamma) \stackrel{\eqref{cocycle}}{=} \\ 
		&\stackrel{\eqref{cocycle}}{=} \int_{G} \left( \int_{S^1} \hat{c}_B(\gamma^{-1}.z, \xi) - \hat{c}_B(\gamma^{-1}.0,\xi)  d \nu(\xi) \right)  d \mu(\gamma)  \stackrel{\eqref{first moment ensures DCT}}{=} \\
		&\stackrel{\eqref{first moment ensures DCT}}{=}  \int_{G} p_\nu(\gamma^{-1}.z) d \mu(\gamma)  - \int_{G}p_\nu(\gamma^{-1}.0)  d \mu(\gamma),
	\end{aligned}
	\]
	and both integrals $\int_{G}p_\nu(\gamma^{-1}.0)  d \mu(\gamma)$ and $\int_{G}p_\nu(\gamma^{-1}.z)  d \mu(\gamma)$ are well-defined due to \eqref{first moment ensures DCT}.
\end{proof}

\textbf{Remark.} Keep in mind that the value $l_{\mu, \nu} := \int_{G}p_\nu(\gamma^{-1}.0)  d \mu(\gamma)$ is a constant which only depends on $\mu$ and $\nu$. Observe that if the subgroup $\Gamma$ generated by the support of $\mu$ is discrete, we obtain the Furstenberg's formula for the drift (see \cite{furstenberg1963noncommuting}), which does not depend on the choice of $\nu$ as well.

Now, let us recall the Fourier expansion of the logarithm of the Poisson kernel:
\begin{equation}
	\label{Fourier series for a familiar log}
	\log(1 - 2x \cos(t) + x^2) = -2 \sum_{k=1}^{\infty} \frac{x^k}{k} \cos(k t)
\end{equation}
where the right series converges uniformly for $|x| < 1$. From this we can deduce
\[
\begin{aligned}
	\hat{c}_B(|z|e^{i \theta}, e^{i t}) &= \log(1 - |z|^2) - \log(1 - 2|z| \cos(\theta - t) + |z|^2) \stackrel{\eqref{Fourier series for a familiar log}}{=} \\ 
								&\stackrel{\eqref{Fourier series for a familiar log}}{=} \log(1 - |z|^2) + 2 \sum_{k=1}^\infty \frac{|z|^k}{k} \cos(k(\theta - t))
\end{aligned}
\]
for all $|z| < 1$. In particular, denoting $\xi = e^{it}$, we can switch from the trigonometric basis to the exponential one to obtain
\begin{equation}
	\label{Fourier series of log pois}
	\begin{aligned}
		\hat{c}_B(z, \xi) &= \log(1 - |z|^2) + \sum_{k=1}^\infty \frac{z^k}{k} e^{-i k t} + \frac{\overline{z}^k}{k} e^{i k t} = \\ 
		&= \log(1 - |z|^2) + \sum_{k=1}^\infty \frac{z^k}{k} \xi^{-k} + \frac{\overline{z}^k}{k} \xi^k
	\end{aligned}
\end{equation}

Now, assume that $\nu \sim \sum_{k \in \mathbb{Z}} a_k e^{i k t}$, where
\begin{equation}
	\label{spectral measure}
	a_k := \frac{1}{2\pi} \int_0^{2 \pi} e^{- i k t} d\nu(t).
\end{equation} 
Combining \eqref{Fourier series of log pois} and \eqref{spectral measure}, we obtain

\begin{equation}
	\label{eq2}
	p_\nu(z) = \log(1 - |z|^2) + \sum_{k=1}^\infty \frac{z^k}{k} a_k + \frac{\overline{z}^k}{k} a_{-k}.
\end{equation}

\textbf{Remark.} The equation \eqref{eq2} proves that $p_\nu(z)$ uniquely defines $\nu$, as we can recover the coefficients by considering

\[
a_k = \frac{1}{(k-1)!} \left. \dfrac{\partial^k}{\partial z^k}  \right|_{z=0} (p_\nu(z) - \log(1 - |z|^2))  , \quad k > 0,
\]
\[
a_{-k} = \frac{1}{(k-1)!} \left. \dfrac{\partial^k}{\partial \overline{z}^k} \right|_{z=0} (p_\nu(z) - \log(1 - |z|^2)), \quad k > 0.
\]
Combining with \eqref{Poisson stat}, we get
\[
\begin{aligned}
	& \left( \int_G \log(1 - |\gamma^{-1}.z|^2) d \mu(\gamma) - \log(1 - |z|^2) \right)  + \\ 
	&+ \left(  \int_G f_+(\gamma^{-1}.z)  d \mu(\gamma) - f_+(z)\right)  + \\ 
	&+ \left( \int_G f_-(\overline{\gamma^{-1}.z})  d \mu(\gamma) - f_-(\overline{z}) \right)  = l_{\mu, \nu},
\end{aligned}
\]
where
\[
\begin{gathered}
	f_+(z) := \sum_{k=1}^\infty \frac{a_k}{k} z^k, \\
	f_-(z) := \sum_{k=1}^\infty \frac{a_{-k}}{k} z^k.
\end{gathered}
\]
Keep in mind that the integrability condition allows us to split the integral. Moreover, we can immediately observe that both $f_+$ and $f_-$ are holomorphic inside $\mathbb{D}$ as $\nu$ is a probability measure, and $|a_k| \le 1$ for all $k \in \mathbb{Z}$.

We can simplify to get
\begin{equation}
	\label{harmonic functions 1}
	\int_G \log(|(\gamma^{-1})'(z)|) d \mu(\gamma) + \int_G \left( f_+(\gamma^{-1}.z) - f_+(z)\right)  d \mu(\gamma) + \int_G \left( f_-(\overline{\gamma^{-1}.z}) - f_-(\overline{z})\right)  d \mu(\gamma) = l_{\mu, \nu}.
\end{equation}

Now we observe that \eqref{harmonic functions 1} features a sum of a harmonic, holomorphic and an antiholomrphic function being constant. Moreover, recall that
\[
\log(|(\gamma^{-1})'(z)|) = Re(\log((\gamma^{-1})'(z))) = \frac{1}{2} (\log((\gamma^{-1})'(z)) + \overline{\log((\gamma^{-1})'(z))})
\]
is precisely the harmonic decomposition of $\log(|(\gamma^{-1})'(z)|)$. We fix the principal branches of the complex logarithm to make this equality unambiguous which is possible due to $\gamma'(z) \ne 0$ for any $\gamma \in PSU(1,1)$ and $|z| < 1$.

Now observe that a sum of a holomorphic function and an antiholomorphic function is constant iff both parts are constant as well, so we get
\begin{equation}
	\frac{1}{2} \int_G \log((\gamma^{-1})'(z))  d\mu(\gamma) + \int_G f_+(\gamma^{-1}.z) d\mu(\gamma) - f_+(z) = l_{\mu, \nu}
\end{equation}
\begin{equation}
	\frac{1}{2} \int_G \log(\overline{(\gamma^{-1})'(z)}) d\mu(\gamma) + \int_G f_-(\overline{\gamma^{-1}.z})d\mu(\gamma) - f_-(\overline{z}) = l_{\mu, \nu}
\end{equation}

These equations are, essentially, the same due to the symmetry $a_k = \overline{a_{-k}}$, as we assume that $\nu$ is positive. Let us redenote $f_+(z) = F_\nu(z)$, so that we are trying to solve
\begin{equation}
	\frac{1}{2} \int_G \log((\gamma^{-1})'(z))  d\mu(\gamma) + \int_G F_\nu(\gamma^{-1}.z)  d\mu(\gamma) - F_\nu(z) = l_{\mu, \nu}.
\end{equation}
Differentiating both sides and making use of the moment condition again to interchange the derivative and the integrals, we can get rid of the logarithm and reduce the equation to
\begin{equation}
	\label{main equation}
	\frac{1}{2} \int_G \frac{(\gamma^{-1})''(z)}{(\gamma^{-1})'(z)}  d\mu(\gamma) + \int_G f_\nu(\gamma^{-1}.z)(\gamma^{-1})'(z)  d\mu(\gamma) - f_\nu(z) = 0,
\end{equation}
where $f_\nu(z) = \sum_{k=0}^{\infty} a_{k+1} z^k$.
Using \eqref{poles of the log derivative}, we can rewrite \eqref{main equation} as follows:
\begin{equation}
	\label{main equation, simp}
	\int_G f_\nu(\gamma^{-1}.z)(\gamma^{-1})'(z)  d\mu(\gamma) - f_\nu(z) = \int_G \frac{d \mu(\gamma)}{z - \gamma.\infty}, \quad |z| < 1.
\end{equation}

Thus, we have proven the first part of Theorem \ref{T:main result}.

\begin{theorem}
	Let $\mu$ be a complex finite Borel measure with the finite first moment with respect to the hyperbolic metric. Consider a probability measure $\nu$ on $S^1$ admitting the Fourier series $\nu \sim \sum_{k \in \mathbb{Z}} a_k z^k$. Then $\nu$ is $\mu$-stationary implies that the function $f_\nu(z) = \sum_{k=0}^\infty a_{k+1} z^k$ satisfies the functional equation \eqref{main equation, simp} for $|z| < 1$.
\end{theorem}


\section{Squeezing water from a stone: a deep dive into \eqref{main equation, simp}}
\label{Squeezing water from a stone}
In this section we will explore the functional equation \eqref{main equation, simp} in much more detail. From now on, we will restrict ourselves to countably supported probability measures $\mu$, denoting by $\Gamma \le G = PSU(1,1)$ the subgroup generated by the support of $\mu$.

\begin{theorem}[Corollary \ref{C:main corollary}.1]
	\label{entire solutions}
	Let $\mu$ be a probability measure with finite support. Then there are no entire solutions to \eqref{main equation, simp}.
\end{theorem}
\begin{proof}
	Choose an element $\tau \in \text{supp}(\mu)$ which does not fix the origin. In particular, $\tau.\infty = (\overline{\tau.0})^{-1} \ne \infty$. Fix small enough contour $C_\tau$ around $\tau.\infty$. Integrating both sides over this contour, we get
	\[
	\begin{aligned}
		& \int_{C_\tau}\left( \int_G \frac{d\mu(\gamma)}{z - \gamma.\infty}  + \int_G f(\gamma^{-1}.z)(\gamma^{-1})'(z)  d\mu(\gamma) - f(z) \right) dz = \\ 
		&= \sum_{\gamma.\infty = \tau.\infty} \mu(\gamma) + \int_\Gamma \int_{\gamma^{-1}(C_\tau)} f(z) dz - \int_{C_\tau} f(z)  = 0
	\end{aligned}
	\]
	by applying the change of variables. As $f(z)$ is entire, the contour integrals vanish, leaving us with $\mu(\gamma) = 0$ for all $\gamma$ with the same pole as $\tau$, which leads to a contradiction.
\end{proof}

\begin{corollary}
	Let $\mu$ be a probability measure with finite support. Then $\limsup_{k \rightarrow \infty} |a_k|^{1/k} > 0$ for every $\mu$-stationary measure with the Fourier series $\nu \sim \sum_{k \in \mathbb{Z}} a_k e^{i k t}$.
\end{corollary}
\begin{proof}
	Consider a $\mu$-stationary measure $\nu$ with $\limsup_{k \rightarrow \infty} |a_k|^{1/k} = 0$. Then $f_\nu(z)$ is an entire function which solves \eqref{main equation, simp} for $|z| < 1$. The LHS of \eqref{main equation, simp} can be analytically continued to a meromorphic function on $\mathbb{C}$, therefore, $f_\nu(z)$ solves \eqref{main equation, simp} for all $\mathbb{C}$. This allows us to apply Theorem \ref{entire solutions}, yielding a contradiction.
\end{proof}

%The next lemma is well-known and straightforward to prove, we will exhibit the proof regardless.
%
%\begin{lemma}
%	\label{T:Cauchy transform}
%	Let $\nu$ be a probability measure on $S^1$. Then we have
%	\begin{equation}
%		\label{E:Cauchy transform}
%		f_\nu(z) = \frac{1}{2 \pi} \int_0^{2 \pi} \dfrac{d \nu(t)}{e^{i t} - z} = C_{e^{-it}\nu}(z).
%	\end{equation}
%	for all $|z| < 1$.
%\end{lemma}
%\begin{proof}
%	It is easy to see that $f_\nu(z)$ is indeed a Cauchy transform, but not of $\nu$ itself: we have to consider a measure $\nu' = e^{-it} \nu$, and then it is true that
%	\[
%	f_\nu(z) = \frac{1}{2 \pi} \int_{0}^{2\pi} \frac{e^{-it} d\nu(t)}{1 - e^{-it} z} = \frac{1}{2 \pi} \int_{0}^{2 \pi} \frac{d\nu(t)}{e^{it} -  z}.
%	\]
%	
%	To show that this equality holds for $|z| < 1$, we can simply expand the geometric series as follows:
%	\[
%	\frac{1}{2 \pi} \int_0^{2 \pi} \dfrac{d \nu(t)}{e^{i t} - z} = \frac{1}{2 \pi} \int_0^{2 \pi} \left( \sum_{k=0}^\infty e^{-i (k+1) t} z^k \right)  d\nu(t) = \sum_{k=0}^\infty \frac{1}{2 \pi} \int_0^{2 \pi} e^{-i (k+1) t} d \nu(t) z^k = f_\nu(z).
%	\]
%\end{proof}

Due to Theorem \ref{T:Smirnov} we know that $f_\nu(z)$ is holomorphic on $\overline{\mathbb{C}} \setminus \mathbb{T}$. Moreover, \eqref{main equation, simp} makes sense for all $z \in \overline{\mathbb{C}}$ outside of the poles of RHS (namely, $\gamma.\infty$ for all $\gamma \in \text{supp} \mu$). 

\begin{theorem}[Theorem \ref{T:main result}]
	\label{T:Cauchy transform solves main eq}
	A probability measure $\nu$ is $\mu$-stationary only if $f_\nu(z)$ solves \eqref{main equation, simp} for all $z \in \overline{\mathbb{C}} \setminus (\mathbb{T} \cup \{\gamma.\infty\}_{\gamma \in \text{supp} \mu} )$.
\end{theorem}
\begin{proof}
	Let $z \in \overline{\mathbb{C}} \setminus (\mathbb{T} \cup \{\gamma.\infty\}_{\gamma \in \text{supp} \mu} )$. Due to Theorem \ref{weighted composition ops} we have
%	The RHS is also holomorphic outside of the unit disk, we invoke Smirnov's theorem [ref].
	\[
	\begin{aligned}
		T_\mu \left( \frac{1}{2 \pi} \int_{0}^{2 \pi} \frac{d\nu(t)}{e^{it} -  z} \right) &= \frac{1}{2 \pi i}  \int_{\mathbb{T}} \left(  \frac{1}{\gamma.e^{it} -  z} - \frac{1}{\gamma.\infty - z} - \frac{1}{e^{it} -  z} \right)  d \nu(t) = \\ 
		&= \frac{1}{2 \pi i}  \int_{\mathbb{T}} \frac{d \gamma_* \nu(w)}{e^{it} -  z}  -  \frac{1}{2 \pi i}\int_{\mathbb{T}} \frac{d \nu(w)}{e^{it} -  z}  - \frac{1}{\gamma.\infty - z} .
	\end{aligned}
	\]
	
	However, as $\nu$ is $\mu$-stationary, we can see that
	\[
	\left( \sum_\Gamma \mu(\gamma) T_{\gamma^{-1}} \right) \left(  \int_{\mathbb{T}} \frac{d \nu(w)}{w(w - z)} \right)  = \int_{\mathbb{T}} \frac{d (\mu * \nu - \nu) (w)}{w(w - z)}  - \int_\Gamma \frac{d \mu(\gamma)}{\gamma.\infty - z} = \int_\Gamma \frac{d \mu(\gamma)}{z - \gamma.\infty},
	\]
	which is precisely \eqref{main equation, simp}.
\end{proof}

\textbf{Remark.} This argument is independent on the proof given in Section \ref{The log-Poisson transform of a stationary measure}.

\begin{theorem}[Corollary \ref{C:main corollary}.2]
	Let $\mu$ be a countably supported probability measure. If \\ $\limsup\limits_{n \rightarrow \infty} \left\| \int_G \frac{d \mu^{*n}(\gamma)}{z - \gamma.\infty} \right\|_1 = \infty$, then there are no $\mu$-stationary measures with $L^{1+\varepsilon}(S^1, Leb)$-density for any $\varepsilon > 0$.
\end{theorem}
\begin{proof}
	Let $\nu$ be $\mu$-stationary with density in $L^{1+\varepsilon}(S^1)$. Due to Fatou's theorem we know that $f_\nu(z) \in H^{1 + \varepsilon}(\mathbb{D})$. In particular, $f_\nu(z) \in H^1(\mathbb{D})$. As all composition operators in LHS of \eqref{main equation, simp} are exactly the operators treated in Theorem \ref{weighted composition ops}, they are isometries, in particular, the $H^1$-norm of LHS is at most $2||f||_1$. Make note of the fact that this application of the triangle inequality does not depend on $\mu$ at all. Applying $H^1$-norm to both sides, we get
	\[
	2 \left\| f_\nu \right\|_1 \ge \left\| \int_G \frac{d \mu(\gamma)}{z - \gamma.\infty} \right\|_1.
	\]
	However, keep in mind that any $\mu$-stationary measure is $\mu^{*n}$-stationary, therefore, WLOG one can replace $\mu$ with $\mu^{*n}$ in the above inequality without changing LHS. This would imply
	\[
	2 \left\| f_\nu \right\|_1 \ge \limsup\limits_{n \rightarrow \infty} \left\| \int_G \frac{d \mu^{*n}(\gamma)}{z - \gamma.\infty} \right\|_1 = \infty,
	\]
	which leads to a contradiction.
\end{proof}

\begin{example}
	Consider $\mu = \delta_\gamma$ for a non-elliptic $\gamma \in PSU(1,1)$. Then the $H^1$-norm of $\frac{1}{z - \gamma^n.\infty}$ goes to infinity as $n \rightarrow \infty$, so there are no absolutely continuous measures with densities in $L^{1 + \varepsilon}(S^1)$, as we expected.
\end{example}

However, as simple as this criterion seems, given a measure $\mu$ supported on a lattice in $PSU(1,1)$, it is not at all easy to estimate $\left\| \int_G \frac{d \mu^*n(\gamma)}{z - \gamma.\infty} \right\|_1$, therefore, a potential argument should rely on how non-uniformly the poles will be distributed in small neighbourhoods of $\mathbb{T}$. Nevertheless, we will make use of this simple idea in Section 5 to deliver the proof of the singularity conjecture.

\subsection{Functional-analytic necessary condition for existence of absolutely continuous stationary measures}

In this subsection we treat LHS of \eqref{main equation, simp} as a bounded operator: define

\[
T_\mu: H^p(\mathbb{D}) \rightarrow H^p(\mathbb{D}), \quad T_\mu(f)(z) := \sum_{\gamma} \mu(\gamma) f(\gamma^{-1}.z)(\gamma^{-1})'(z) - f(z).
\]

It is well-known that $T_\mu$ is a bounded operator for all $0 < p \le \infty$, and in such generality, not much else is known about $T_\mu$. If $p > 1$, we can at least explicitly compute its adjoint $T^*_\mu : H^q \rightarrow H^q$.

\begin{proposition}
	Consider $V_\gamma(f)(z) = f(\gamma^{-1})(\gamma^{-1})'(z)$ as a bounded operator $H^p \rightarrow H^p$. Then
	\[
	V^*_\gamma(f)(z) = S^*(f(\gamma.z) \gamma.z), \quad f \in H^q(\mathbb{D}),
	\]
	where $S^*$ stands for the backwards shift $S^*(g)(z) = \frac{g(z) - g(0)}{z}$.
\end{proposition}
\begin{proof}
	As in many similar computations (see \cite[Theorem 2]{cowen1988linear} for an example), we use the reproducing kernel property of $\frac{1}{1 - \overline{a}z}$: for any $f \in H^p$
	\[
	\left\langle f(z), \frac{1}{1 - \overline{a}z} \right\rangle := \frac{1}{2\pi} \int_{0}^{2 \pi} \frac{f(e^{it}) dt}{1 - a \overline{z}} = f(a).
 	\]
 	A slight modification yields
 	\[
 	\left\langle f(z), \frac{1}{a - z} \right\rangle = \frac{f(\overline{a}^{-1})}{\overline{a}}.
 	\]
 	As we know how $T_\gamma$ acts on $\frac{1}{a - z}$, reflexivity of $H^p$ for all $1 < p < \infty$ allows us to write
 	\[
 	\begin{aligned}
 		\frac{V^*_\gamma f(\overline{a}^{-1})}{\overline{a}} &= \left\langle (V_\gamma^*)f(z), \frac{1}{a - z} \right\rangle =  \left\langle f(z), \frac{1}{\gamma.a - z} - \frac{1}{\gamma.\infty - z} \right\rangle = \\ 
 		&=\frac{f(\overline{\gamma.a}^{-1})}{\overline{\gamma.a}} - \frac{f(\overline{\gamma.\infty}^{-1})}{\overline{\gamma.\infty}} \stackrel{\eqref{from inside to outside}}{=} \gamma.\overline{a^{-1}} f(\gamma.\overline{a^{-1}}) - \gamma.0 f(\gamma.0).
 	\end{aligned}
 	\]
 	Replacing $\overline{a}^{-1}$ with $\omega$, we get
 	\[
 	V^*_\gamma f(w) = \frac{\gamma.w f(\gamma.w) - \gamma.0 f(\gamma.0)}{w} = S^*(f(\gamma.w) \gamma.w).
 	\]
\end{proof}

As a quick corollary, we get that

\begin{equation}
	\label{adjoint of LHS}
	T_\mu^*(f)(z) = S^*\left( \sum_{\gamma} \mu(\gamma) f(\gamma.z) \gamma.z\right)  - f(z).
\end{equation}

\begin{theorem}
	Let $\mu$ satisfy the Blaschke condition. Then for any $\mu$-stationary measure $\nu$ with $L^p$-density for $1 < p < \infty$, we have
	\[
	f_\nu(z) \in (\overline{T_\mu^*(B_\mu H^q)})^{\perp} \subset H^p,
	\]
	where $B_\mu(z)$ is the Blaschke function corresponding to the support of $\mu$:
	\[
	B_\mu(z) := \prod_{\gamma \in \text{supp}(\mu)} \gamma^{-1}(z).
	\]
\end{theorem}

\begin{proof}
	Let $T_\mu(f) = \sum_{\gamma} \frac{\mu(\gamma)}{z - \gamma.\infty}$. It is easy to see that $\sum_{\gamma} \frac{\mu(\gamma)}{z - \gamma.\infty}$ is a linear combination of reproducing kernels $\frac{1}{1 - \overline{\gamma.0}z}$. In particular, $\sum_{\gamma} \frac{\mu(\gamma)}{z - \gamma.\infty} \in (B_\mu H^q)^\perp$. Therefore,
	\[
	0 = \left\langle T_\mu(f), B_\mu H^q \right\rangle = \left\langle f, \overline{T_\mu^*(B_\mu H^q)} \right\rangle.
	\]
\end{proof}

This proves Corollary \ref{functional-analytic necessary condition}. 

%Recall the Douglas-Shields-Shapiro theorem:
%
%\begin{theorem}[Douglas-Shields-Shapiro, $(p>1)$-version]
%	Let $1 < p < \infty$ and consider $f \in H^p$. Then the following are equivalent:
%	\begin{enumerate}
%		\item $f$ is non-cyclic with respect to $S^*$, that is, $
%		\text{span}\{ (S^*)^k(f) \}_{k \ge 0}$ is \textbf{not} dense in $H^p(\mathbb{D})$.
%		\item There exists a holomorphic function $\varphi \in H^2$ with $|\varphi(e^{it})| = 1$ a.e. on $S^1$ (inner function) such that $f \in (\varphi H^q)^\perp$.
%		\item There exists an inner function $\varphi \in H^2$ such that $f / \varphi$ admits a pseudocontinuation to a function $\tilde{f} \in H^p_0(\mathbb{D}_e)$.
%	\end{enumerate} 
%\end{theorem}
%
%As a corollary from this theorem, we can deduce the following:
%
%\begin{corollary}
%	Let $\mu$ satisfy the Blaschke condition, and assume that $\overline{T_\mu^*(B_\mu H^q)} = \varphi H^q$ for some inner $\varphi \in H^2$. Then every solution to \eqref{main equation, simp} is pseudocontinuable.
%\end{corollary}
%\begin{proof}
%	Beurling's theorem implies that $\overline{T_\mu^*(B_\mu H^q)}$ is invariant with respect to the forward shift $f(z) \mapsto zf(z)$. Therefore, the orthogonal complement is $S^*$-invariant in $H^p$. Douglas-Shields-Shapiro theorem applies, and we get that $f$ is pseudocontinuable.
%\end{proof}
%
%Keep in mind that any Cauchy transform of a singular measure is pseudocontinuable, but the converse it not true, therefore, the above corollary is still a much weaker statement. Nevertheless, we can still make use of this observation: suppose that $\nu$ is an absolutely continuous $\mu$-stationary measure. As the operator $T_\mu$ takes pseudocontinuations to pseudocontinuations, and RHS itself has both inner and outer non-tanegntial limits which coincide a.e., we would get two distinct (meromorphic!) solutions of \eqref{main equation, simp} on $\mathbb{D}_e$: one comes from the Cauchy transform itself, and another from the previous Corollary. This would cause the operator $T_\mu$ to have a meromorphic function on $\mathbb{D}_e$ in its kernel, which seems unlikely in the discrete case. From what we understand, the injectivity of $T_\mu$ is still an open question.
%
%
%To conclude this subsection, we want to remind the reader that the difficulty of this problem lies precisely in the fact that $T_\mu$ does \textbf{not} commute with the backward shift $S^*$, which disallows us from easily arguing that $T_\mu$ preserves the $S^*$-invariant subspaces in any way.
\subsection{When the Lebesgue measure is stationary?}

Earlier we have reproved the well-known fact that the Lebesgue measure cannot be a stationary measure if $\mu$ has finite support. To understand this case better, we need to look at \eqref{main equation, simp} and observe that $f_\nu$ vanishes, leaving us with vanishing of the following \textbf{Borel series}.
\begin{equation}
	\label{Lebesgue equation}
	\sum_{\gamma} \frac{\mu(\gamma)}{z - \gamma.\infty} = 0, \quad |z| < 1.
\end{equation}
This immediately proves Corollary \ref{C:main corollary}.1. At first glance, it might seem counter-intuitive that the above sum can vanish on the entire disc, but recall the following fundamental fact about Borel series.


\begin{theorem}[\cite{brownsums}, Theorem 3]
	\label{non-tangential equiv rep}
	Let $A = \{ z_n \} \subset \mathbb{D}$ be a sequence of points \textbf{inside} the unit disk without interior limit points. Then there exists a sequence $\{c_n\} \in l^1$ such that
	\[
	\sum_n \frac{c_n}{z - z_n} = 0, \quad |z| > 1
	\]
	if and only if almost every point in $S^1$ is a non-tangential limit of a subsequence in $\{z_n\}$.
\end{theorem}

This theorem almost gives what we want, however, the above theorem gives series with poles \textbf{inside} the disk which vanishes \textbf{outside} of it, whereas we need the opposite -- Borel series with poles \textbf{outside} the disk and vanishing \textbf{inside} the disk.

One can easily mitigate this by considering the change of variables $z \mapsto 1/z$:
\[
\begin{aligned}
	\sum_{\gamma} \frac{\mu(\gamma)}{z^{-1} - \gamma.\infty} &= \sum_{\gamma} \frac{\mu(\gamma) z}{1 - (\gamma.\infty) z} = \sum_{\gamma} \frac{(\gamma.\infty)^{-1}  \mu(\gamma) z}{(\gamma.\infty)^{-1} -  z} = \\ 
	&= \sum_{\gamma} - \frac{\mu(\gamma)}{\gamma.\infty} + \frac{\mu(\gamma)}{(\gamma.\infty)^2} \frac{1}{(\gamma.\infty)^{-1} - z}.
\end{aligned}
\]
However, as we can plug in $z = 0$ in \eqref{Lebesgue equation}, we get that
\[
\sum_{\gamma} \frac{\mu(\gamma)}{z^{-1} - \gamma.\infty} = \sum_{\gamma}\frac{\mu(\gamma)}{(\gamma.\infty)^2} \frac{1}{(\gamma.\infty)^{-1} - z} = 0
\]
for all $|z| > 1$. Applying the Brown-Shields-Zeller theorem, we obtain Corollary \ref{intro: Lebesgue is stationary}.3.

\textbf{Remark.} Recall that the orbit of a point with respect to an action of a discrete subgroup of $PSU(1,1)$ is non-tangentially dense if and only if the subgroup is of the first type. Therefore, Theorem \ref{non-tangential equiv rep} confirms that $\Gamma \subset PSU(1,1)$ being a first-kind Fuchsian group should be a necessary condition for a Furstenberg measure on $\Gamma$ to exist.

Moreover, due to another theorem of Beurling, referring to \cite[Corollary 4.2.24]{Shapiro1968}:

\begin{theorem}[\cite{beurling1934fonctions}, \cite{beurling1989collected}]
	Let $\{ z_n \}$ be a sequence of points \textbf{outside} of the unit disk with $|z_n| \downarrow 1$. If $\limsup\limits_{n \rightarrow \infty} |c_n|^{1/n} < 1$ and
	\[
	\sum_n \frac{c_n}{z - z_n} = 0, \quad |z| < 1,
	\]
	then all $c_n = 0$.
\end{theorem}
Applying this theorem to $z_n = \gamma_n.\infty$ (relative to a suitable enumeration of $\Gamma$), we get that a Fuchsian group of first kind $\Gamma \subset PSU(1,1)$ admits a Furstenberg measure only if
\[
\limsup\limits_{n \rightarrow \infty} |\mu(\gamma_n)|^{1/n} = 1,
\] 
thus proving Corollary \ref{intro: Lebesgue is stationary}.2. Combined with the exponential growth of Fuchsian groups, this condition implies that a Furstenberg measure $\mu$ cannot have a double-exponential moment with respect to the hyperbolic distance: if we let $c > 0$, then
\[
\sum_{n} \mu(\gamma_n) e^{e^{c d(0, \gamma_n.0)}} < \infty \iff \sum_{n} \mu(\gamma_n) e^{c n} < \infty \Rightarrow \limsup\limits_{n \rightarrow \infty} |\mu(\gamma_n)|^{1/n} < e^{-c} < 1.
\]
As for the strongest known moment conditions: it is known that J.Li's counterexample, given in the Appendix of \cite{10.1215/00127094-2020-0058}, provides a Furstenberg measure with an exponential moment, our approach shows that a Furstenberg measure cannot have a double-exponential moment. It is widely believed that the approach developed in \cite{connellmuchnik} should yield an example of a Furstenberg measure with a superexponential moment, but we are not aware of a complete and self-contained argument being published.

Finally, we would like to remark that the proof of \cite[Theorem 3]{brownsums} is, essentially, non-constructive. In context of our problem, the idea is as follows.

\begin{enumerate}
	\item We start by considering an operator $H^\infty(\mathbb{D}) \rightarrow l^\infty(\Gamma)$, 
	\[
	T(f)_\gamma = f((\gamma.\infty)^{-1}).
	\]
	\item Its image is proven to be $wk^*$-closed in $l^\infty(\Gamma)$.
	\item Consider $p = \delta_{e} \in l^\infty(\Gamma)$. A clever argument shows that \\ $\text{dist}(p, T(H^\infty(\mathbb{D}))) = \frac{1}{2}$, so the image cannot coincide with the whole $l^\infty(\Gamma)$. and applying the Hahn-Banach theorem, we prove the existence of an element $a \in l^1(\Gamma)$, such that:
	\begin{itemize}
		\item for any $y \in T(H^\infty(\mathbb{D})) \subset l^\infty(\Gamma)$ we have $y(a) = 0$
		\item $p(a) = 1$
		\item $||a|| < 2 + \varepsilon$ for a small enough $\varepsilon > 0$.
	\end{itemize}
	\item The sequence $a$ solves our problem, that is,
	\[
	\sum_{\gamma} \dfrac{a_\gamma}{z - (\gamma.\infty)^{-1}} = 0.
	\]
\end{enumerate}

\section{The proof of the singularity conjecture}
\label{The proof of the singularity conjecture}
\subsection{A motivating example}
We will start this section with a simple but an instructive example.
\begin{example}
	Consider a random walk $(X_n)$ generated by one single atom supported on a non-elliptic loxodromic element $\gamma \in PSU(1,1)$. Suppose we were to ``prove'' that the hitting measure for this random walk is singular without actually knowing in advance that it is atomic. Even in this elementary case we know that $X_n$ converges to $S^1$ almost surely, so, on the level of Cauchy-Szeg\H{o} transforms, we know that 
	\[
	\mathbb{E}\left[\frac{1}{X_n.\infty - z}\right] \rightarrow \mathbb{E}\left[\frac{1}{X_\infty.\infty - z}\right]
	\]
	on compact subsets of $\mathbb{D}$. However, because the random walk is so simple, we can explicitly write
	\[
	\mathbb{E}\left[\frac{1}{X_n.\infty - z}\right] = \frac{1}{\gamma^n.\infty - z},
	\]
	and it is not difficult to prove (see \cite[Lemma 6.2.23]{cimahardy}) that
	\[
	\mathbb{E}\left[\frac{1}{X_n.\infty - z}\right] = \frac{1}{\gamma^n.\infty - z} \xrightarrow{H^p} \frac{1}{\gamma_+ - z} = \mathbb{E}\left[\frac{1}{X_\infty.\infty - z}\right].
	\]
	Moreover, it is also not difficult to work out that
	\[
	\cfrac{1}{\cfrac{\gamma^n.\infty}{|\gamma^n.\infty|} - z} \xrightarrow{H^p} \frac{1}{\gamma_+ - z},
	\]
	but $\cfrac{1}{\cfrac{\gamma^n.\infty}{|\gamma^n.\infty|} - z} \in H^p \cap \overline{H^p_0}$, so the limit has to be in $H^p \cap \overline{H^p_0}$ as well.
\end{example}

This section is dedicated to showing that the very same phenomenon holds for \textbf{any} admissible random walk with the finite first moment on a non-elementary discrete subgroup of $PSU(1,1)$, yielding a complete classification of $\mu$-stationary measures in the discrete case.

\subsection{Main argument}


Let $\mu$ be an admissible probability measure with the finite first moment on a non-elementary discrete subgroup $\Gamma \subset PSU(1,1)$ generating a random walk which we will denote by $(X_n)$. We will define two sequences of functions

\begin{gather}
	g_n(z) := \mathbb{E} \left[\frac{1}{X_n.\infty - z}\right] = \mathbb{E} \left[\frac{\overline{X_n.0}}{1 - \overline{X_n.0} z}\right], \\
	h_n(z) := \mathbb{E} \left[\frac{1}{\frac{X_n.\infty}{|X_n.\infty|} - z}\right] = \mathbb{E} \left[\frac{\overline{X_n.0}}{|X_n.0| - \overline{X_n.0} z}\right].
\end{gather}

Because almost surely $X_n.0$ converges to a point $X_\infty \in S^1$, by taking inverses, we see that $X_n.\infty$ converges to $S^1$ almost surely as well. Via a slight abuse of notation, let us denote this limit by $X_\infty.\infty$. Therefore,
\[
\lim\limits_{n \rightarrow \infty} \mathbb{E} \left[\frac{1}{X_n.\infty - z}\right] = \mathbb{E} \left[\frac{1}{X_\infty.\infty - z}\right] = f_\nu(z)
\]
for any $|z| < 1$. The convergence can be trivially upgraded to convergence on compact subsets of $\overline{\mathbb{C}} \setminus \mathbb{T}$, and this fact precisely corresponds to the $wk^*$-convergence of $\mu^{*n}$ to the hitting measure $\nu$. Of course, this does not show singularity by itself, but we aim to show that the rational approximation induced by $g_n$ converges in the stronger $H^p$-topology for all $p \in (0, 1)$, ``forcing'' singularity by pinning down $f_\nu(z)$ into an exotic subspace $H^p \cap \overline{H^p_0}$, which has trivial intersections with $H^{1 + \varepsilon}$ for any $\varepsilon > 0$ and $p \in (0,1)$ due to the Riesz-Riesz theorem (see \cite[Theorem 1]{Ale79} for original argument, or more accessible \cite[Corollary 6.2.12]{cimahardy}). 

\begin{theorem}
	\label{convergence in H^p}
	Let $(X_n)$ be an admissible random walk on a non-elementary discrete subgroup $\Gamma \leq PSU(1,1)$ with the finite first moment. Then for every $0 < p < 1$ the following statements hold:
	\begin{enumerate}
		\item $g_n(z) \xrightarrow{H^p} f_\nu(z)$.		
		\item $\lim\limits_{n \rightarrow \infty} d_{H^p}(g_n, h_n) = 0$.
	\end{enumerate}
\end{theorem}

Before we prove the theorem, we need the following standard lemma that is commonly used to work with $H^p$-spaces for $p \in (0, 1)$.

\begin{lemma}[\cite{boundedgarnett}]
	\label{Hp estimate positive real}
	Let $f(z) \in \mathcal{H}(\mathbb{D})$ with $Re(f(z)) > 0$ for all $z \in \mathbb{D}$. Then for all $p \in (0, 1)$ we have
	\[
	\left\| f \right\|^p_p \le A_p |f(0)|^p,
	\]
	where $A_p = 1 / \cos(p \pi / 2)$. 
\end{lemma}

\begin{proof}[Proof of Theorem \ref{convergence in H^p}]
	\indent
	\begin{itemize}
		\item 
		
		Let $\delta > 0, \theta \in \mathbb{R}$. In order to get the necessary estimate, we need to explicitly keep track of the probability space on which $X_n : \Omega \rightarrow \Gamma$ and the limit $X_\infty.\infty: \Omega \rightarrow \mathbb{T}$ are defined. Define an event 
		\[
		\begin{aligned}
			E_{\theta, \delta, N}  := & \{ \omega \in \Omega : |X_\infty.\infty - e^{i \theta}| < \delta \} \cup \\  \cup & \{\omega \in \Omega : | X_n.\infty - e^{i \theta}| < 2 \delta \  \forall n > N\} \cup \\ \cup & \{\omega \in \Omega : \exists n > N \  X_n.0 = 0 \}.
		\end{aligned}		
		\]
		
		Consider two integrals:
		\begin{gather}
			\left| \int_{\Omega \setminus E_{\theta, \delta, N}} \frac{1}{X_n.\infty - e^{i \theta}} - \frac{1}{X_\infty.\infty - e^{i \theta}} d \mathbb{P}(\omega) \right| \\
			\left| \int_{E_{\theta, \delta, N}} \frac{1}{X_n.\infty - e^{i \theta}} - \frac{1}{X_\infty.\infty - e^{i \theta}} d \mathbb{P}(\omega) \right|.
		\end{gather}
		We will proceed by estimating these two integrals in two different ways. The first integral can be treated by noticing that the random walk conditioned on the complement to $E_{\theta, \delta}$ almost surely misses a neighbourhood $e^{i \theta}$ starting from $n = N$:
		\[
		\begin{aligned}
			& \left| \int_{\Omega \setminus E_{\theta, \delta, N}} \frac{1}{X_n.\infty - e^{i \theta}} - \frac{1}{X_\infty.\infty - e^{i \theta}} d \mathbb{P}(\omega) \right| \le
			\int_{\Omega \setminus E_{\theta, \delta, N}} \left| \frac{1}{X_n.\infty - e^{i \theta}} - \frac{1}{X_\infty.\infty - e^{i \theta}} \right| d \mathbb{P}(\omega) = \\ 
			& = \int_{\Omega \setminus E_{\theta, \delta, N}} \frac{|X_\infty.\infty - X_n.\infty|}{|X_n.\infty - e^{i \theta}| |X_\infty.\infty - e^{i \theta}|} d \mathbb{P}(\omega) \le \frac{1}{2 \delta^2} \int_{\Omega \setminus E_{\theta, \delta, N}} |X_\infty.\infty - X_n.\infty| d \mathbb{P}(\omega)
		\end{aligned}
		\]
		for any $n > N$. A straightforward application of DCT yields that as $n \rightarrow \infty$, the expectation $\int_{\Omega \setminus E_{\theta, \delta, N}} |X_\infty - X_n| d \mathbb{P}(\omega)$ goes to zero.
		
		As for the second integral, we observe that the probability of $E_{\theta, \delta, N}$ goes to zero as $\delta \rightarrow 0$, because the hitting measure is non-atomic. So, we can separate the integral like this:
		
		\[
		\left| \int_{E_{\theta, \delta, N}} \frac{1}{X_n.\infty - e^{i \theta}} - \frac{1}{X_\infty.\infty - e^{i \theta}} d \mathbb{P}(\omega) \right|^p \le \left| \int_{E_{\theta, \delta, N}} \frac{1}{X_n.\infty - e^{i \theta}} d \mathbb{P}(\omega) \right|^p + \left| \int_{E_{\theta, \delta, N}} \frac{1}{X_\infty.\infty - e^{i \theta}} d \mathbb{P}(\omega) \right|^p .
		\]
		Then we use Lemma \ref{convergence in H^p} to show that after integrating with respect to $\theta$, these integrals will turn out to be small.
		Now we proceed to estimate the $H^p$-norms themselves. Fix $\delta, N$. Then for every $n > N$ we have
		\[
		\begin{aligned}
			& \left\| \mathbb{E} \left[\frac{1}{X_n.\infty - z} - \frac{1}{X_\infty.\infty - z}\right]  \right\|_p^p = \int_0^{2 \pi} \left| \int_{\Omega} \frac{1}{X_n.\infty - e^{i \theta}} - \frac{1}{X_\infty.\infty - e^{i \theta}} d \mathbb{P}(\omega) \right|^p \frac{d \theta}{2 \pi} = \\
			&= \int_0^{2 \pi} \left| \int_{\Omega \setminus E_{\theta, \delta, N}} \frac{1}{X_n.\infty - e^{i \theta}} - \frac{1}{X_\infty.\infty - e^{i \theta}} d \mathbb{P}(\omega) + \int_{E_{\theta, \delta, N}} \frac{1}{X_n.\infty - e^{i \theta}} - \frac{1}{X_\infty.\infty - e^{i \theta}} d \mathbb{P}(\omega) \right|^p \frac{d \theta}{2 \pi} \le \\
			& \le \int_0^{2 \pi} \left| \int_{\Omega \setminus E_{\theta, \delta, N}} \frac{1}{X_n.\infty - e^{i \theta}} - \frac{1}{X_\infty.\infty - e^{i \theta}} d \mathbb{P}(\omega) \right|^p + \left| \int_{E_{\theta, \delta, N}} \frac{1}{X_n.\infty - e^{i \theta}} - \frac{1}{X_\infty.\infty - e^{i \theta}} d \mathbb{P}(\omega) \right|^p \frac{d \theta}{2 \pi}.
		\end{aligned}
		\]
		Exploiting the above estimates, we get
		\[
		\begin{aligned}
			& \int_0^{2 \pi} \left| \int_{\Omega \setminus E_{\theta, \delta, N}} \frac{1}{X_n.\infty - e^{i \theta}} - \frac{1}{X_\infty.\infty - e^{i \theta}} d \mathbb{P}(\omega) \right|^p + \left| \int_{E_{\theta, \delta, N}} \frac{1}{X_n.\infty - e^{i \theta}} - \frac{1}{X_\infty.\infty - e^{i \theta}} d \mathbb{P}(\omega) \right|^p \frac{d \theta}{2 \pi} \le \\ 
			& \le \frac{1}{2^p\delta^{2p}} \int_0^{2 \pi} \left( \int_{\Omega \setminus E_{\theta, \delta, N}} |X_\infty.\infty - X_n.\infty| d \mathbb{P}(\omega) \right)^p \frac{d \theta}{2 \pi} + \\ &+ \left\| \int_{E_{\theta, \delta, N}} \frac{1}{X_n.\infty - z} d \mathbb{P}(\omega) \right\|_p^p + \left\| \int_{E_{\theta, \delta, N}} \frac{1}{X_\infty.\infty - z} d \mathbb{P}(\omega) \right\|_p^p \stackrel{L \ref{Hp estimate positive real}}{\le}\\
			&\le \frac{1}{2^p\delta^{2p}}  \left( \int_{\Omega \setminus E_{\theta, \delta, N}} |X_\infty.\infty - X_n.\infty| d \mathbb{P}(\omega) \right)^p + A_p \left(  \left| \int_{E_{\theta, \delta, N}} (X_n.0) d \mathbb{P}(\omega) \right|^p + \left| \int_{E_{\theta, \delta, N}} (X_\infty.0) d \mathbb{P}(\omega) \right|^p \right) \le \\ & \le \frac{1}{2^p\delta^{2p}} \left( \int_{E_{\theta, \delta, N}} |X_\infty.\infty - X_n.\infty| d \mathbb{P}(\omega) \right)^p + 2 A_p \mathbb{P}(E_{\theta, \delta, N})^p
		\end{aligned}
		\]
		First of all, we choose such $\delta> 0$ and $N \in \mathbb{N}$ so that $\mathbb{P}(E_{\theta, \delta, N})^p$ is small enough, then we increase $N$ until the first integral becomes small.
		\item The idea is mostly the same, we can just repeat the same estimates. Define
		
		\[
		\begin{aligned}
			G_{\theta, \delta, N} := &\left\lbrace \omega \in \Omega : \left| \frac{X_n.\infty}{|X_n.\infty|} - e^{i \theta} \right| < \delta \ \forall n > N \right\rbrace \cup \\ \cup & \{ \omega \in \Omega : | X_n.\infty - e^{i \theta} | < \delta \ \forall n > N \} \cup \\ \cup & \{\omega \in \Omega : \exists n > N \  X_n.0 = 0 \}.
		\end{aligned}
		\]
		
		Using the same argument as above, for every $n > N$ (we fix $\delta > 0, N > 0$ beforehand) we get
		
		\[
		\begin{aligned}
			& \left\| \mathbb{E} \left[\frac{1}{X_n.\infty - z} - \frac{1}{\frac{X_n.\infty}{|X_n.\infty|} - z}\right] \right\|_p^p = \int_0^{2 \pi} \left| \int_{\Omega} \frac{1}{X_n.\infty - e^{i \theta}} - \frac{1}{\frac{X_n.\infty}{|X_n.\infty|} - e^{i \theta}} d \mathbb{P}(\omega) \right|^p \frac{d \theta}{2 \pi} \le \\ 
			&\le \frac{1}{\delta^{2 p}} \left| \int_{\Omega \setminus G_{\theta, \delta, N}} \left|\frac{X_n.\infty}{|X_n.\infty|} - X_n.\infty \right| d \mathbb{P}(\omega) \right|^p + A_p \left(  \left| \int_{G_{\theta, \delta, N}} (X_n.0) d \mathbb{P}(\omega) \right|^p + \left| \int_{G_{\theta, \delta, N}} \frac{X_n.0}{|X_n.0|} d \mathbb{P}(\omega) \right|^p \right) \\ 
			& \le \frac{1}{\delta^{2 p}} \left| \int_{\Omega \setminus G_{\theta, \delta, N}} \left|\frac{X_n.\infty}{|X_n.\infty|} - X_n.\infty \right| d \mathbb{P}(\omega) \right|^p + 2 A_p \mathbb{P}(G_{\theta, \delta, N})^p.
		\end{aligned}
		\]
		Again, a.s. convergence makes the first integral small, and the second term is controlled by $\mathbb{P}(G_{\theta, \delta, N})^p$.
	\end{itemize}
\end{proof}
\begin{corollary}
	Let $(X_n)$ be an admissible random walk on a non-elementary discrete subgroup $\Gamma \leq PSU(1,1)$ with the finite first moment. If $\nu$ is the $\mu$-stationary measure, then $f_\nu(z) \in H^p \cap \overline{H^p_0}$ for all $0 < p < 1$. In particular, there are only three possible outcomes:
	\begin{enumerate}
		\item $\nu$ is the Lebesgue measure,
		\item $\nu$ admits an $L^1(S^1)$-density which does not belong to $L^{(1+\varepsilon)}(S^1)$ for any $\varepsilon > 0$,
		\item $\nu$ is singular with respect to the Lebesgue measure.
	\end{enumerate}
\end{corollary}
\begin{proof}
	The above theorem implies that $(h_n)_{n > 0}$ is also a Cauchy sequence in $H^p$. However, due to Aleksandrov we know that
	\[
	\overline{span}^{H^p} \left\lbrace \frac{1}{1 - e^{i\lambda} z} \right\rbrace_{\lambda \in \mathbb{R}} = H^p \cap \overline{H^p_0},
	\]
	and
	\[
	h_n \in \overline{span}^{H^p} \left\lbrace \frac{1}{1 - e^{i\lambda} z} \right\rbrace_{\lambda \in \mathbb{R}},
	\]
	therefore, $\lim\limits_{n \rightarrow \infty} h_n = f_\nu \in H^p \cap \overline{H^p_0}$. The conclusion follows from the Fatou's jump theorem and the $(p > 1)$-version of \cite[Corollary 6.2.12]{cimahardy} which follows from that Theorem \ref{T: Riesz-Riesz theorem}. Therefore, $(H^p \cap \overline{H^p_0}) \cap H^{1 + \varepsilon} = \{0\}$. 
\end{proof}

\begin{theorem}
	Let $(X_n)$ be a admissible random walk on a non-elementary discrete subgroup $\Gamma \leq PSU(1,1)$ supported on a set satisfying the Blaschke condition with the finite first moment. Then the hitting measure $\nu$ is either singular with respect to the Lebesgue measure or it admits an $L^1(S^1)$-density which does not belong to $L^{(1+\varepsilon)}(S^1)$ for any $\varepsilon > 0$. If $(X_n)$ has a finite superexponential moment then $\nu$ is singular.
\end{theorem}
\begin{proof}
	As the support satisfies the Blaschke condition, the RHS of \eqref{main equation, simp} cannot be identically zero due to the coherence property, see \cite[Proposition 4.2.14]{Shapiro1968}. Therefore, $f(z) = 0$ is not the solution to \eqref{main equation, simp}, so Lebesgue measure is not stationary. Assuming superexponential moment, \cite[Theorem 1.5(iii)]{blachere2011harmonic} implies that if $\nu$ is absolutely continuous, then the density is in $L^\infty(S^1)$. In particular, it belongs to $L^{1 + \varepsilon}(S^1)$, and the previous corollary shows that this is impossible.
\end{proof}


\section{Open questions}
\begin{itemize}
	\item The question of whether the converse of Theorem \ref{T:main result} holds is slightly harder than one might anticipate. It is easy to see from \eqref{eq2} that the integral transform $\nu \mapsto p_\nu(z)$ is injective. In particular, we get that $\text{span} \left\lbrace  \xi \mapsto \log\left( \frac{1 - |z|^2}{|z - \xi|^2} \right) : z \in \mathbb{D} \right\rbrace $ is dense in $L^2(S^1, m)$, but we aim for a stronger statement -- we want this span to be dense in $C(S^1)$. This would allow us to use the Riesz-Markov-Kakutani theorem in order to establish that $\nu$ is $\mu$-stationary. Stone-Weierstrass theorem does not apply as the space of functions respect to which stationarity holds does not need to be a subalgebra of $C(S^1)$.
	\item It is easy to see from the proof of Corollary \ref{C:main corollary}.1 that we actually get non-existence of solutions $f(z) = \sum a_{k+1} z^k$ to \eqref{main equation, simp} with $\limsup_{n \rightarrow \infty} |a_k|^{1/k} < \varepsilon$ for some small $\varepsilon$, as only one preimage of the chosen contour explodes, so we can bound the radius of the convergence of the solution. Ideally, one would like to show that for finitely supported $\mu$ every solution of \eqref{main equation, simp} has radius of convergence exactly $1$. Keep in mind that this result would almost close the smoothness gap: it is known that absolutely continuous densities stationary with respect to finitely supported measures can belong to $C^n(S^1)$ for any $n > 1$. 
	
	The Douglas-Shields-Shapiro theorem implies that any holomorphic function with the radius of convergence exceeding $1$ is either cyclic with respect to the backward shift or rational. It is reasonable to assume that \eqref{main equation, simp} only has rational solutions when $\mu$ is supported on a single element, and we conjecture that former never happens.
	\item The Brown-Shields-Zeller theorem has an unexpected consequence -- it requires the poles to be non-tangentially dense \textbf{almost} everywhere on $S^1$. Therefore, even if $\Gamma$ is a non-cocompact lattice, there will be a sequence $(a_\gamma) \in l^1(\Gamma)$ such that
	\[
	\sum \frac{a_\gamma}{z - \gamma.\infty} = 0, \quad |z| < 1.
	\]
	However, due to \cite{guivarch1990} we know that the Lebesgue measure is not stationary with respect to any $\mu$ with finite first moment. Therefore, either Theorem \ref{T:main result} is not a criterion, $(a_\gamma)$ does not have the first finite moment, or there is a complex-valued Furstenberg measure -- keep in mind that Guivarch'-le Jan's methods only apply for probability measures $\mu$.
	\item Cauchy transforms were used to study affine self-similar measures on $\mathbb{C}$ in \cite{lund1998cauchy}. However, the paper was focused on studying measures supported on fractals with the Hausdorff dimension $\alpha> 1$, which forces the Cauchy transforms to be bounded and Holder with exponent $\alpha - 1$ (see \cite[Theorem 2.1(b)]{lund1998cauchy}). It should be possible to characterize the Hausdorff dimension of hitting measures using a similar self-similarity condition for the non-tangential limit $f_\nu(e^{it})$, but we expect the Hausdorff dimensions be strictly smaller than $1$, and we are not aware of any existing ideas in this direction.
\end{itemize}

\printbibliography
\Addresses
\end{document}
