\documentclass[11pt]{amsart}

\usepackage[utf8]{inputenc}
\usepackage{amsmath, amsfonts, amssymb, amsthm, cases, csquotes, enumitem, hyperref, graphicx, mathrsfs, mathtools, tikz, tikz-cd}
\usepackage[style=alphabetic]{biblatex}

\addbibresource{singularity.bib}
\theoremstyle{definition}

\newtheorem{definition}{Definition}[section]
\newtheorem{theorem}{Theorem}[section]
\newtheorem{proposition}{Proposition}[section]
\newtheorem{corollary}{Corollary}[section]
\newtheorem{lemma}{Lemma}[section]
\newtheorem{example}{Example}[section]
\newtheorem{conjecture}{Conjecture}[section]

%opening
\title{Measures at infinity (and beyond!)}
\author{Petr Kosenko}

%\usepackage[left=3cm,right=3cm,top=3cm,bottom=3cm,bindingoffset=0cm]{geometry}

\begin{document}
	\maketitle
	\begin{abstract}
		This is a collaborative effort in order to collect all existing results related to the singularity conjecture and oultine existing approaches to its resolution.
	\end{abstract}
	
	\section{Introduction}
	\subsection{Goals}
	This survey is intended to be an up-to-date survey on the singularity conjecture, serving as an improved version of a brief survey outlined in the author's PhD thesis. These are our main objectives:
	\begin{itemize}
		\item First of all, we want to clarify the origins behind the singularity conjecture, giving proper attributions once and for all.
		\item Secondly, we aim to collect all results related to the singularity conjecture, taking the Furstenberg's theorem about Poisson boundaries of random walks on lattices in semisimple Lie groups as a starting point. In particular, we want to properly highlight which cases are solved and which are still open.
		\item Finally, we want to discuss the higher-dimensional and higher-rank generalizations of the singularity conjecture as well.
	\end{itemize}
	
	\section{Notation and preliminary definitions}
		\subsection{Random walks on groups}
			\begin{definition}
				A \textbf{random walk} on a group $\Gamma$ generated by a probability measure $\mu$ and initial distribution $\mu_0$ is the sequence $(X_n)_{n \ge 0}$, where 
				\[
				X_n = X_0 g_1 \dots g_n,
				\]
				where $X_0, g_i$ are independent $\Gamma$-valued random variables, where $X_0$ is $\mu_0$-distributed, and $g_i$ are $\mu$-distributed.
			\end{definition}
			\textbf{Remark.} As usual, we assume that $\mu_0$ is fully supported on the identity, in other words, unless explicitly stated, random walks start from the identity.
	
			Let us recall the following important quantities:
			\begin{definition}
				Let $(X_n)$ be a random walk on a group $\Gamma$ generated by a probability measure $\mu$. Let $x, y \in \Gamma$. Then we define the following quantities:
				\begin{itemize}
					\item The \textbf{first-passage function} is $F_\mu(x, y) := \mathbb{P}(\exists n : X_n = x^{-1} y)$.
					\item The \textbf{Green function} $G_\mu(x, y) := \sum_{n=0}^{\infty} \mathbb{P}(X_n = x^{-1} y)$.
					\item The \textbf{Green distance} is defined as follows: $d_\mu(x, y) := - \log(F_\mu(x, y))$.
				\end{itemize} 
			\end{definition}
			
	\section{Singularity conjecture}
	
	Here we provide the statement of the \textbf{singularity conjecture}.
	\begin{conjecture}[ \cite{MR2568439}, (\cite{kaimanovich2011matrix}, page 259)]
		\label{Fuchsian singularity conjecture}
		For any finitely supported measure $\mu$ on $\text{SL}_2(\mathbb{R})$, whose support generates a discrete subgroup, the harmonic/hitting/stationary measure for the random walk driven by $\mu$ is singular with respect to the Lebesgue measure.
	\end{conjecture}
	
	\textbf{Remark.} The exact origins of this conjecture are difficult to track down, as there were, apparently, no written accounts of the conjecture before 2008. We were able to get in touch with V. Kaimanovich and V. Kleptsyn, and, in accordance to their accounts of the origins behind the singularity conjecture, we refer to \cite{MR2568439} and \cite{kaimanovich2011matrix} as primary sources for the statement itself. In line with the expositions provided in these references, we attribute the singularity conjecture to H. Furstenberg, Y. Guivarc'h, V. Kaimanovich, F. Ledrappier and V. le Prince. 
	
	The existing results related to Conjecture \ref{Fuchsian singularity conjecture} can be categorized as follows:
	
	\begin{itemize}
		\item Historically, the first results were \textbf{negative}, where dropping a certain condition from the singularity conjecture would yield an example of a random walk with absolutely continuous harmonic measure. A \textbf{positive} result would consist of establishing the singularity of the harmonic measure for a class of random walks on some lattices in $\text{SL}_2(\mathbb{R})$.
		\item 
	\end{itemize}
	
	\subsection{Negative results}
	\subsubsection{Dropping finiteness of the support}
	The very first result related to Conjecture \ref{Fuchsian singularity conjecture} is negative!
	
	\begin{theorem}[\cite{furstenberg71}, Theorem 4.1, Theorem 5.1]
		Let $G = \mathrm{SL}_d(\mathbb{R})$ for $d \ge 2$. Consider a lattice $\Gamma \subset G$. If we denote the space of upper-triangular $d \times d$-matrices by $U$, then there exists a probability measure $\mu$ on $\Gamma$ such that the Poisson boundary of $(\Gamma, \mu)$ is isomorphic to $(B, \nu)$, where $B$ is the homogeneous space $G / U$, and $\nu$ is the unique $O(d)$-invariant measure on $B$. 
	\end{theorem} 
	
	Moreover, this construction yields a measure $\mu$ on $\Gamma$ with an infinite support and the finite first moment with respect to the \textbf{hyperbolic} distance. Such counterexamples are pretty general, as \cite{connelmuchnik} shows:
	
	\begin{theorem}[\cite{connelmuchnik}, Theorem 0.3]
		If $\Gamma$ is a group acting geometrically on a $CAT(-1)$-space $X$, and $\nu$ is a H\"{o}lder $\alpha$-quasiconformal measure on $\partial X$, then there exists a measure $\mu$ on $\Gamma$ such that $\nu$ is a harmonic measure with respect to $\mu$.
	\end{theorem}
	\textbf{Remark.} The author is not aware of the moment conditions which are guaranteed by Connel-Muchnik construction.
	
	Inspired by \cite{furstenberg71}, \cite{connelmuchnik}, the authors of \cite{linaudpan} were able to show that the Furstenberg's construction can be specified to convex cocompact subgroups of $G = \text{SL}_2(\mathbb{R})$ (as we will mention later, this construction works for $G = \text{SO}_+(1, n)$ as well) yielding counterexamples with the finite \textbf{exponential} moment with respect to the \textbf{hyperbolic} distance.
	
	An upcoming preprint by M. Mirzakhani, A. Eskin, K. Rafi and W. Pan uses the methods of \cite{connelmuchnik} in order to construct a similar counterexample for any \textbf{hyperbolic} group. As of May 27, 2023, it has not been announced yet.
	
	\subsubsection{Dropping discreteness of the subgroup}
	In this short subsection we want to remark that the statement of the conjecture in \cite{kaimanovich2011matrix} omitted discreteness, however, by the time of publication, the authors were already aware of \cite{Bourgain2012} and \cite{MR2969625}, which (independently, to the author's knowledge) provide an example of a finite-range random walk on a \textbf{dense} subgroup of $\text{SL}_2(\mathbb{R})$ with the absolutely continuous harmonic measure.
	
	Some of the latest results regarding this case are presented in (as of May 27, 2023 unpublished) \cite{kogler2022local} and \cite{kittle2023absolutely}.
		
	\subsection{Positive results}
	
	As the techniques used for cocompact Fuchsian groups and non-cocompact groups are vastly different, we would like to separate these two cases.
	
	\subsubsection{Non-cocompact groups}
	
	The first positive result was achieved by Y. Guivarc'h and Y. le Jan:
	
	\begin{theorem}[\cite{guivarch1990}]
		\label{IntT: singular for non-cocompact}
		Let $\Gamma$ be a cofinite non-cocompact Fuchsian group equipped with a finite-range random walk generated by a probability measure $\mu$. Then, identifying the Poisson boundary of $\Gamma$ with $S^1$, the harmonic measure $\nu$ is singular with respect to the Lebesgue measure. 
	\end{theorem}
	
	There is no satisfactory enough but brief explanation of the techniques, but, roughly speaking, the idea is to distinguish the ``winding'' of a random geodesic in $\Gamma \setminus \mathbb{H}^2$ with respect to both the hitting measure and with respect to the uniform measures. This result was strengthened in \cite{MR2568439} in 
	\cite{Gadre2015WordLS} for random walks with finite first moment with respect to the \textbf{word} metric, the methods used there are different from \cite{guivarch1990}.
	
	Moreover, in \cite{MR4069238} this result was strengthened as follows:
	
	\begin{theorem}[\cite{MR4069238}]
		Let $\Gamma$ be a finitely generated subgroup containing a parabolic element, and consider an admissible random walk with a finite superexponential moment. Then the harmonic measure is singular with respect to the Lebesgue measure.
	\end{theorem}
	
	The most general result was announced (not published as of May 31, 2023) in \cite{benard2023winding}, removing the superexponential moment and admissibility conditions. 
	
	\begin{theorem}[\cite{benard2023winding}, Corollary 1.3]
		Let $\Gamma$ be a finitely generated subgroup containing a parabolic element. Let $\mu$ be a non-elementary probability measure on $\Gamma$ with finite first moment with respect to a \textbf{word} metric. Then the harmonic measure is singular with respect to the Lebesgue measure.
	\end{theorem}
	
	Therefore, \cite{benard2023winding} and \cite{linaudpan} fully settle the non-cocompact case:
	
	\begin{theorem}[\cite{benard2023winding}, \cite{linaudpan}]
		Let $\Gamma$ be finitely generated non-cocompact subgroup of $\text{SL}_2(\mathbb{R})$.
		\begin{itemize}
			\item If $\Gamma$ contains a parabolic element, then for every random walk with finite first moment with repsect to the word metric the harmonic measure is singular.
			\item If $\Gamma$ is convex cocompact, then there always exists a random walk with a finite exponential moment such that the harmonic measure is singular with respect to the Patterson-Sullivan measure on the limit set.
		\end{itemize}
	\end{theorem}
	
	\subsubsection{Cocompact groups} 
	
	The above mentioned papers all exploit the existence of a cusp is a significant way, so these methods do not immediately transfer to cocompact groups. 
	
	The first positive results were obtained for cocompact groups corresponding to regular tesselations, in \cite
	
	
	\section{Generalizations}
	\subsection{Higher-dimensional singularity conjecture}
	\subsection{Higher-rank singularity conjecture}
	
	\printbibliography
	
\end{document}