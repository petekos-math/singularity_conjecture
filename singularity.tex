\documentclass[11pt]{amsart}

\usepackage[utf8]{inputenc}
\usepackage{amsmath, amsfonts, amssymb, amsthm, cases, csquotes, enumitem, hyperref, graphicx, mathrsfs, mathtools, tikz, tikz-cd}
\usepackage[style=alphabetic]{biblatex}

\addbibresource{singularity.bib}
\theoremstyle{definition}

\newtheorem{definition}{Definition}[section]
\newtheorem{theorem}{Theorem}[section]
\newtheorem{proposition}{Proposition}[section]
\newtheorem{corollary}{Corollary}[section]
\newtheorem{lemma}{Lemma}[section]
\newtheorem{example}{Example}[section]
\newtheorem{conjecture}{Conjecture}[section]

%opening
\title{Measures at infinity (and beyond!)}
\author{Petr Kosenko}

%\usepackage[left=3cm,right=3cm,top=3cm,bottom=3cm,bindingoffset=0cm]{geometry}

\begin{document}
	\maketitle
	\begin{abstract}
		This is a collaborative effort in order to collect all existing results related to the singularity conjecture and oultine existing approaches to its resolution. If you want to contribute, email me at \email{pk226575@gmail.com} or \email{pkosenko@math.ubc.ca}.
	\end{abstract}
	
	\section{Introduction}
	\subsection{Goals}
	This survey is intended to be an up-to-date survey on the singularity conjecture, serving as an improved version of a brief survey outlined in the author's PhD thesis. These are our main objectives:
	\begin{itemize}
		\item First of all, we want to clarify the origins behind the singularity conjecture, giving proper attributions once and for all.
		\item Secondly, we aim to collect all results related to the singularity conjecture, taking the Furstenberg's theorem about Poisson boundaries of random walks on lattices in semisimple Lie groups as a starting point. In particular, we want to properly highlight which cases are solved and which are still open.
		\item Finally, we want to discuss the higher-dimensional and higher-rank generalizations of the singularity conjecture as well.
	\end{itemize}
	\textbf{Remark.} Initially, we did not plan for this preprint to go beyond outlining existing results related to singularity of harmonic measures. However, we strongly feel that the fact that the singularity conjecture will require substantial interdisciplinary effort due to its deep connections with several difficult problems in geometric group theory, probability theory, and homogeneous dynamics. For this reason we will try to include possible directions in which the conjecture and its generalizations could be attacked from.
	
	\section{Acknowledgements}
	We would like to acknowledge the (direct and indirect) contrubutions of Giulio Tiozzo, Steven Cantrell, Steven Lalley, Tianyi Zheng, Wenyu Pan, Kunal Chawla, Ilya Gekhtman, Mark Pollicott, Polina Vytnova, Tushar Das, and Sebastien Gouezel, among many others.
	
	\section{Notation and preliminary definitions}
		\subsection{Random walks on groups}
			\begin{definition}
				A \textbf{random walk} on a countable group $\Gamma$ generated/induced by a probability measure $\mu$ is the sequence $(X_n)_{n \ge 0}$, where 
				\[
				X_n = g_1 \dots g_n,
				\]
				where $g_i$ are independent $\Gamma$-valued random variables, and $g_i$ are $\mu$-distributed.
			\end{definition}
			\textbf{Remark.} As usual, we assume that the initial distribution is fully supported on the identity, in other words, unless explicitly stated, random walks start from the identity. Thus, we will often define a random walk by a pair $(\Gamma, \mu)$, withour mentioning the random variables $X_n$ explicitly.
	
			Let us recall the following important quantities:
			\begin{definition}
				Let $(X_n)$ be a random walk on a countable group $\Gamma$ generated by a probability measure $\mu$. Let $x, y \in \Gamma$. Then we define the following quantities:
				\begin{itemize}
					\item The \textbf{first-passage function} is $F_\mu(x, y) := \mathbb{P}(\exists n : X_n = x^{-1} y)$.
					\item The \textbf{Green function} $G_\mu(x, y) := \sum_{n=0}^{\infty} \mathbb{P}(X_n = x^{-1} y)$.
					\item The \textbf{Green distance} is defined as follows: $d_\mu(x, y) := - \log(F_\mu(x, y))$.
				\end{itemize} 
			\end{definition}
			
			For this survey it will be important to keep in mind random walks satsfying the following \textbf{moment conditions}:
			\begin{definition}
				Consider a random walk $(X_n)$ induced by a probability measure $\mu$ on a group $G$ equipped with a left-invariant distance $d(g, h)$. We say that $(X_n)$ (or $\mu$) has finite
				\begin{itemize}
					\item \textbf{logarithmic moment} with respect to $d$, if
					\[
					\sum_{g \in \Gamma} \mu(g) \max \{ \log(d(e, g)), 0 \} < \infty.
					\]
					\item \textbf{first moment} with respect to $d$, if
					\[
					\sum_{g \in \Gamma} \mu(g) d(e, g) < \infty.
					\]
					\item \textbf{exponential moment} with respect to $d$, if there exists a constant $a > 0$ such that
					\[
					\sum_{g \in \Gamma} \mu(g) e^{a d(e, g)} < \infty.
					\]
					\item \textbf{superexponential moment} with respect to $d$, if
					\[
					\sum_{g \in \Gamma} \mu(g) e^{a d(e, g)} < \infty.
					\]
					for every $a > 0$.
				\end{itemize}
			\end{definition}
			
		\subsection{The visibility compactification and Furstenberg boundary for semi-simple Lie groups}
		%TODO I would rather have an actual expert in homogeneous dynamics write this section...		
		Let $G$ be a connected semi-simple real Lie group with finite center. Consider its maximal compact subgroup $K$, and let $S = G / K$ be the corresponding Riemannan symmetric space with the fixed origin $o \simeq K$. As the Cartan involution induces a decomposition $\mathfrak{g} = \mathfrak{k} \oplus \mathfrak{p}$, we can identify the tangent space of $S$ with $\mathfrak{p}$, and by restricting the Killing from from $\mathfrak{g}$ to $\mathfrak{p}$ we obtain the Riemannian manifold structure on $S$.
		
		A choice of a maximal abelian subalgebra $\mathfrak{a} \subset \mathfrak{g}$ and its open Weyl chamber $\mathfrak{a}_+$ defines the \textbf{polar/Cartan} decomposition $G = K\exp(\overline{\mathfrak{a}_+}) K$. As the exponential map $\exp : \mathfrak{p} \rightarrow S$ is a diffeomeorphism, every geodesic ray in $S$ starting from $o$ can be represented by $k \exp(a t) o$ for some uniquely determined $a \in \overline{\mathfrak{a}_+}$, and $t \in \mathbb{R}$. 
		
		This gives rise to the visibility compactification $\partial S = T^1_o(S)$, where we identify the unit sphere in the tangent space with geodesic rays originating at $o \in S$. The visibility boundary comes equipped with the action of $G$ itself, where for every $g \in G$ we define $g.(k \exp(a t) o)$ as the unique geodesic ray $g_k \exp(a t) o$ starting from $g o$ asymptotic to the original ray.
		
		This construction immediately suggests that the action is \textbf{not} transitive for higher-rank ($\dim \, \mathfrak{a} > 1$) Lie groups since the $G$-orbits do correspond to the unit vectors in $\overline{\mathfrak{a}_+}$. Finally, each orbit $\partial S_a$ corresponding to interior unit elements $a \in \mathfrak{a}_+$ is isomorphic to the space of \textbf{full flags} $\mathcal{B}$, whereas orbits corresponding to unit vectors in $\overline{\mathfrak{a}_+} - \mathfrak{a}_+$ correspond to suitable quotients of quotients of $\mathcal{B}$.  
			
		\subsection{Hitting measures}
		For the purposes of this survey we will require the following classical results about the convergence of random walks on countable groups to their various boundaries.
		\begin{theorem}[\cite{woess_2000}]
			\label{general convergence to Martin}
			Let $\Gamma$ be a countable group equipped with a probability measure $\mu$. Then the random walk $(X_n)(\omega)$ almost surely converges to a random variable $X_\infty(\omega)$ in the Martin boundary $\partial_M \Gamma$.
		\end{theorem}
		\begin{theorem}[\cite{kaimanovich2000poisson}, Theorem 7.6, Theorem 7.7]
			\label{hyperbolic convergence to Gromov}
			Let $\mu$ be a probability measure on a hyperbolic group $\Gamma$ with a support generating (as a semigroup) a non-elementary subgroup. Then almost surely a sample path $(X_n(\omega))$ converges to the Gromov boundary, and the pushforward $\nu$ of $\mathbb{P}_\mu$ via the resulting map $\Gamma^\mathbb{N} \rightarrow \partial \Gamma$ is a unique $\mu$-stationary probability measure on $\partial \Gamma$, which is called the \textbf{hitting measure}. 
			
			Moreover, if $\mu$ has a finite logarithmic moment with respect to a word distance $d_w$ and finite entropy $H(\mu) := - \sum_{g \in \Gamma} \log(\mu(g)) \mu(g) < \infty$, then the Poisson boundary of $(\Gamma, \mu)$ is isomorphic to $(\partial \Gamma, \nu)$.
		\end{theorem}
		\begin{theorem}[\cite{kaimanovich2000poisson}, Theorem 10.3]
			\label{higher-rank convergence to Furstenberg}
			Let $\mu$ be a probability measure on a discrete subgroup $\Gamma \subset G$ of a semi-simple Lie group $G$ with a finite first moment with respect to the Riemannian distance in $S$: $\sum_{g \in \Gamma} \text{dist}(o, g.o) \mu(g) < infty$. 
			\begin{enumerate}
				\item Almost every sample path $X_n(\omega)$ of the random walk $(G, \mu)$ is regular, that is, there exists a geodesic ray $\xi$ and a number $l > 0$ such that $d(X_n(\omega), \xi(nl)) = o(n)$. 
				\item For almost every sample path $X_n(\omega)$ the limit $\lim\limits_{n \rightarrow \infty} a(X_n(\omega))/n \in \text{SL}_n(\mathbb{R}) / \text{U}_n(\mathbb{R})$ for $x_0$ exists.
				\item If $a \ne 0$, then for almost every sample path $X_n(\omega)$ the sequence $X_n(\omega).o$ converges in the visibility compactification to a limit point from the orbit $\partial S_a$.
				\item If $a = 0$, then the Poisson boundary of the pair $(\Gamma, \mu)$ is trivial, and if $a \ne 0$, then it is isomorphic to $(\partial S_a, \nu)$, where $\nu$ is the hitting measure corresponding to $\mu$. (it exists due to the previous point)
			\end{enumerate} 
		\end{theorem}
		For $n = 2$ the Furstenberg boundary is isomorphic to $S^1$, and the symmetric space $\text{SL}_2(\mathbb{R}) / \text{SO}(2)$ is just $\mathbb{H}^2$.
		
		\subsection{Structure theorem for Fuchsian groups}
		Recall that a discrete subgroup of $\text{SL}_2(\mathbb{R})$ is called a \textbf{Fuchsian group}.
		
		\begin{definition}
			Let $\Gamma$ be a Fuchsian group equipped with a natural action on $\mathbb{H}^2$. Its \textbf{limit set} $\Lambda(\Gamma) \subset S^1$ is the set of all limit points of all orbits $\Gamma z$ for some $z \in \mathbb{H}^2$. 
		\end{definition}
		\begin{theorem}
%			TODO put the Borthwick reference
			\label{classification of Fuchsian}
			Let $\Gamma$ be a Fuchsian group. Then the limit set either:
			\begin{enumerate}
				\item is empty ($\Gamma$ is finite)
				\item consists of a single element ($\Gamma$ is generated by a parabolic element)
				\item consists of two elements ($\Gamma$ is generated by a hyperbolic element)
				\item coincides with $S^1$ (a first kind Fuchsian group)
				\item is a perfect nowhere dense compact subset of $S^1$ (a second kind Fuchsian group) 
			\end{enumerate}
		\end{theorem}
		
		For the purposes of our survey, we will only care about finitely generated Fuchsian groups of first kind.
		
		
		\begin{proposition}
%			TODO put the Beardon reference here
			\label{classification of first kind}
			Let $\Gamma$ be a finitely generated Fuchsian group of a first kind. Then $\Gamma$ is geometrically finite (\cite[Theorem 4.6.1]{book:291162}), and has finite volume. Therefore, any such group is either
			\begin{enumerate}
				\item cocompact, that is, $\Gamma$ admits a compact fundamental domain,
				\item cofinite non-cocompact.
			\end{enumerate} 
		\end{proposition}
		\textbf{Hint:} prove that any fundamental polygon has all its sides strictly inside $\mathbb{H}^2$.
			
			
	\section{Singularity conjecture}
	
	Here we provide the statement of the \textbf{singularity conjecture}.
	\begin{conjecture}[ \cite{MR2568439}, (\cite{kaimanovich2011matrix}, page 259)]
		\label{Fuchsian singularity conjecture}
		For any finitely supported measure $\mu$ on $\text{SL}_2(\mathbb{R})$, whose support generates a discrete subgroup, the harmonic/hitting/stationary measure for the random walk driven by $\mu$ is singular with respect to the Lebesgue measure.
	\end{conjecture}
	
	\textbf{Remark.} The exact origins of this conjecture are difficult to track down, as there were, apparently, no written accounts of the conjecture before 2008. We were able to get in touch with V. Kaimanovich and V. Kleptsyn, and, in accordance to their accounts of the origins behind the singularity conjecture, we refer to \cite{MR2568439} and \cite{kaimanovich2011matrix} as primary sources for the statement itself. In line with the expositions provided in these references, we attribute the singularity conjecture to H. Furstenberg, Y. Guivarc'h, V. Kaimanovich, F. Ledrappier and V. le Prince. 
	
	Let us start with trivial remarks: the conjecture is trivially true if the subgroup $\Gamma$ generated by the support of $\mu$ is not of first kind, as the hitting measure has to be supported on $\Lambda(\Gamma)$. Therefore, due to Proposition \ref{classification of first kind}, it is enough to restrict ourselves to cofinite Fuchsian groups. Thus, the existing results related to Conjecture \ref{Fuchsian singularity conjecture} can be categorized as follows:
	
	\begin{itemize}
		\item Historically, the first results were \textbf{negative}, where dropping a certain condition from the singularity conjecture would yield an example of a random walk with absolutely continuous harmonic measure. A \textbf{positive} result would consist of establishing the singularity of the harmonic measure for a class of random walks on some lattices in $\text{SL}_2(\mathbb{R})$.
		\item The approaches for cocompact and non-cocompact groups differ significantly, so it makes sense to separate these cases as well.
	\end{itemize}
	
	\textbf{Remark.} If we replace the Lebesgue measure with
	
	\subsection{Negative results}
	\subsubsection{Dropping finiteness of the support}
	The very first result related to Conjecture \ref{Fuchsian singularity conjecture} is negative!
	
	\begin{theorem}[\cite{furstenberg71}, Theorem 4.1, Theorem 5.1]
		Let $G = \mathrm{SL}_d(\mathbb{R})$ for $d \ge 2$. Consider a lattice $\Gamma \subset G$. If we denote the space of upper-triangular $d \times d$-matrices by $U$, then there exists a probability measure $\mu$ on $\Gamma$ such that the Poisson boundary of $(\Gamma, \mu)$ is isomorphic to $(B, \nu)$, where $B$ is the homogeneous space $G / U$, and $\nu$ is the unique $O(d)$-invariant measure on $B$. 
	\end{theorem} 
	
	Moreover, this construction yields a measure $\mu$ on $\Gamma$ with an infinite support and the finite first moment with respect to the \textbf{hyperbolic} distance. Such counterexamples are pretty general, as \cite{connelmuchnik} shows:
	
	\begin{theorem}[\cite{connelmuchnik}, Theorem 0.3]
		If $\Gamma$ is a group acting geometrically on a $CAT(-1)$-space $X$, and $\nu$ is a H\"{o}lder $\alpha$-quasiconformal measure on $\partial X$, then there exists a measure $\mu$ on $\Gamma$ such that $\nu$ is a harmonic measure with respect to $\mu$.
	\end{theorem}
	\textbf{Remark.} The author is not aware of the moment conditions which are guaranteed by Connel-Muchnik construction.
	
	Inspired by \cite{furstenberg71}, \cite{connelmuchnik}, the authors of \cite{linaudpan} were able to show that the Furstenberg's construction can be specified to convex cocompact subgroups of $G = \text{SL}_2(\mathbb{R})$ (as we will mention later, this construction works for $G = \text{SO}_+(1, n)$ as well) yielding counterexamples with the finite \textbf{exponential} moment with respect to the \textbf{hyperbolic} distance.
	
	An upcoming preprint by M. Mirzakhani, A. Eskin, K. Rafi and W. Pan uses the methods of \cite{connelmuchnik} in order to construct a similar counterexample for any \textbf{hyperbolic} group. As of May 27, 2023, it has not been announced yet.
	
	\subsubsection{Dropping discreteness of the subgroup}
	In this short subsection we want to remark that in the announced version, the statement of the conjecture in \cite{kaimanovich2011matrix} omitted discreteness, however, by the time of publication, the authors were already aware of \cite{Bourgain2012} and \cite{MR2969625}, which (independently, to the author's knowledge) provide an example of a finite-range random walk on a \textbf{dense} subgroup of $\text{SL}_2(\mathbb{R})$ with the absolutely continuous harmonic measure.
	
	Some of the latest results regarding this case are presented in (unpublished, as of May 27, 2023) \cite{kogler2022local} and \cite{kittle2023absolutely}.
		
	\subsection{Positive results}
	
	As the techniques used for cocompact Fuchsian groups and non-cocompact groups are vastly different, we would like to separate these two cases.
	
	\subsubsection{Non-cocompact groups}
	
	The first positive result was achieved by Y. Guivarc'h and Y. le Jan:
	
	\begin{theorem}[\cite{guivarch1990}]
		\label{IntT: singular for non-cocompact}
		Let $\Gamma$ be a cofinite non-cocompact Fuchsian group equipped with a finite-range random walk generated by a probability measure $\mu$. Then, identifying the Poisson boundary of $\Gamma$ with $S^1$, the harmonic measure $\nu$ is singular with respect to the Lebesgue measure. 
	\end{theorem}
	
	There is no satisfactory enough but brief explanation of the techniques, but, roughly speaking, the idea is to distinguish the ``winding'' of a random geodesic in $\Gamma \setminus \mathbb{H}^2$ with respect to both the hitting measure and with respect to the uniform measures. This result was strengthened in \cite{MR2568439} and 
	\cite{Gadre2015WordLS} for random walks with finite first moment with respect to the \textbf{word} metric, the methods used there are different from \cite{guivarch1990}.
	
	Moreover, in \cite{MR4069238} this result was strengthened as follows:
	
	\begin{theorem}[\cite{MR4069238}]
		Let $\Gamma$ be a finitely generated subgroup containing a parabolic element, and consider an admissible random walk with a finite superexponential moment. Then the harmonic measure is singular with respect to the Lebesgue measure.
	\end{theorem}
	
	The most general result was announced (not published as of May 31, 2023) in \cite{benard2023winding}, removing the superexponential moment and admissibility conditions. 
	
	\begin{theorem}[\cite{benard2023winding}, Corollary 1.3]
		Let $\Gamma$ be a finitely generated subgroup containing a parabolic element. Let $\mu$ be a non-elementary probability measure on $\Gamma$ with finite first moment with respect to a \textbf{word} metric. Then the harmonic measure is singular with respect to the Lebesgue measure.
	\end{theorem}
	
	Unifying the statements in \cite{benard2023winding} and \cite{linaudpan}, we get the following.
	
	\begin{theorem}[\cite{benard2023winding}, \cite{linaudpan}]
		Let $\Gamma$ be finitely generated subgroup of $\text{SL}_2(\mathbb{R})$.
		\begin{itemize}
			\item If $\Gamma$ contains a parabolic element, then for every random walk with finite first moment with repsect to the word metric the harmonic measure is singular.
			\item If $\Gamma$ does not contain a parabolic element, then there always exists a random walk with a finite exponential moment and infinite support such that the harmonic measure is absolutely continuous with respect to the Patterson-Sullivan measure on the limit set.
		\end{itemize}
	\end{theorem}
	
	Therefore, the hitting measures can detect whether $\Gamma$ contains a parabolic element.
	
	\textbf{(an unfinished) Exercise.} Let $\Gamma = \text{SL}_2(\mathbb{Z}) \subset \text{SL}_2(\mathbb{R})$. 
	\begin{enumerate}
		\item Consider 
		\[
		T = \begin{pmatrix} 1 & 1 \\ 0 & 1 \end{pmatrix}, \quad S = \begin{pmatrix} 0 & 1 \\ -1 & 0 \end{pmatrix}.
		\]
		Show that $S$ and $T$ generate $\Gamma$.
		\item Establish a group isomorphism $\mathbb{Z}_2 * \mathbb{Z}_3 \simeq \Gamma$.
		
		\textbf{Hint:} consider $x = S^2$, $y = (ST)^2$. 
		\item Consider $\mu = \frac{1}{4} (\delta_T + \delta_{T^{-1}}) + \dfrac{1}{2} \delta_S$. This measure generates a simple random walk $X_n = g_1 \dots g_n$ on $\Gamma$. Show that the $\lim\limits_{n \rightarrow \infty} X_n$ (convergence to the Gromov boundary of $\Gamma$) exists almost surely without using the Kaimanovich's theorem.
		
		\textbf{Hint:} with respect to $x, y$, the convergence in the free product is easy to show. Can we leverage the quasi-isometry between the Cayley graphs with respect to different generators to get this? 
		\item Realize each infinite geodesic as a word $T^{m_1} S T^{m_2} S \dots$, where $m_1 \in \mathbb{Z}$ and $m_i \ne 0$ for all $i > 2$. To each such word correspond a (backwards) continued fraction as follows:
		\[
		T^{m_1} S T^{m_2} S T^{m_3} S \dots \mapsto m_1 - \cfrac{1}{m_2 - \cfrac{1}{m_3 - \cfrac{1}{\dots}}} \in \mathbb{R}.
		\]
		
		Show that this correspondence respects the action of $\Gamma$ on $\mathbb{R}$. For more details about see \cite{series}.
		
		\item Find asymptotics for the hitting measures of the intervals $[n, n+1]$ for $n \in \mathbb{Z}$. Compare with the measure induced from $S^1$.
		\item Prove the singularity conjecture for $(\Gamma, \mu)$.
		\item Adapt the same arguemnt for a different set of generators:
		\[
		L = \begin{pmatrix} 1 & 1 \\ 0 & 1 \end{pmatrix}, \quad R = \begin{pmatrix} 1 & 0 \\ 1 & 1 \end{pmatrix}.
		\]
	\end{enumerate}
	\textbf{Question:} is there a general argument which works for any support in $\text{SL}_2(\mathbb{Z})$? Probably, not...
	
	\subsubsection{Cocompact groups} 
	
	The above mentioned papers all exploit the existence of a cusp is a significant way, so these methods do not immediately transfer to cocompact groups. 
	
	The first positive results were obtained independently for simple random walks on cocompact groups corresponding to regular tessellations, in \cite{Carrasco2019OnTS}, and for arbitrary symmetric nearest-neighbour random walks in \cite{10.1093/imrn/rnaa213}.
	
	\begin{theorem}[\cite{Carrasco2019OnTS}, \cite{10.1093/imrn/rnaa213}]
		Consider a group $\Gamma$ generated by the hyperbolic translations $(t_i)_{i = 1, \dots, n}$ identifying the opposite sides of a regular hyperbolic polygon with the internal angles $\dfrac{2\pi}{m}$. Then for every \textbf{symmetric} measure $\mu$ supported on $t_i$, the hitting measure is singular with respect to the Lebesgue measure (except for exceptional pairs oultined below.)  
	\end{theorem}
	
	For simple random walks we can explicitly estimate the Furstenberg-type integrals to get the estimates for the Hausdorff dimension (see \cite{Carrasco2019OnTS}), or we can use Theorem \ref{BHMTheorem 1.4} to use similar (in spirit) estimates for the translation lengths of the generators to show that they dominate the Green distance via passing to a covering random walk on the respective free group. (see \cite{10.1093/imrn/rnaa213}.)
	
	\textbf{Remark.} While estimating the 
	
	\textbf{Open cases:} As of June 07, 2023, we know that the singularity conjecture holds for symmetric nearest-neighbour random walks on Fuchsian groups generated by side-pairing translations, identifying the opposite sides of $\Delta_{n, m}$, with three exceptions: $(n, m) = (4, 5), (8, 3), (10, 3)$. It is highly desirable to find some elements with large translation lengths in these three tessellations.
	
	Moreover, the approach breaks for arbitrary pairings. However, it is easy to show that the conjecture still holds, if we restrict ourselves to simple random walks, but relax the pairing condition. (exercise!)
	
	The result in \cite{10.1093/imrn/rnaa213} was generalized to nearest-neighbour random walks on hyperelliptic cocompact groups in \cite{kosenko_tiozzo_2022}.
	
	In an unpublished preprint \cite{kosenko2023asymptotics} we show that we can push this result further for random walks on hyperelliptic cocompatc groups supported on the powers of the side-pairing translations.
 	
	\subsection{Miscellaneous results}
	\begin{theorem}[\cite{blachere2011harmonic}, Theorem 1.5]
		\label{BHMTheorem 1.4}
		Let $\Gamma$ be a non-elementary hyperbolic group, and suppose that $d$ is a hyperbolic left-invariant distance on $\Gamma$. Let $d_\varepsilon$ be a visual metric on $\partial \Gamma$, and $\nu$ is the harmonic measure given by a symmetric probability measure $\mu$ on $\Gamma$ with an exponential moment, the support of which generates $\Gamma$. Also suppose that $(\Gamma, d_\mu)$ is a hyperbolic group as well. By $\rho$ we will denote a quasiconformal measure on $(\partial \Gamma, d_\varepsilon)$. Then the following are equivalent.
		\begin{enumerate}
			\item $h_{\mu} \le l_{d, \mu} v_d$.
			\item $\rho$ and $\nu$ are equivalent.
			\item $\rho$ and $\nu$ are equivalent, and the density is almost surely bounded, and bounded away from 0.
			\item The map $(\Gamma, d_\mu) \xrightarrow{Id} (\Gamma, vd)$ is a $(1, C)$-quasi-isometry. Equivalently,
			\[
			\sup_{g \in \Gamma} |d_\mu(e, g) - vd(e, g)| < \infty.
			\]
			\item The measure $\nu$ is a quasiconformal measure of $(\partial \Gamma, d_\varepsilon)$.
		\end{enumerate}
	\end{theorem}
	This results directly works only for symmetric RW's on cocompact Fuchsian groups, and, arguably, provides the most promising approach to a complete resolution of the singularity conjecture.
	
	Moreover, due to \cite[Theorem 1.1]{cantrell2023manhattan}, we know that Green metrics are, in a certain sense, ``dense'' in the space of all left-invariant pseudo-metrics on a hyperbolic group $\Gamma$.
	
	\section{Generalizations}
	\subsection{Singularity conjecture for Fuchsian groups of second kind}
	As we have mentioned earlier, the singularity conjecture, as stated, does hold trivially for Fuchsian groups of second kind. Nevertheless, in the spirit of Theorem \ref{BHMTheorem 1.4}, it makes sense to ask whether the harmonic measure is equivalent to the \textbf{Patterson-Sullivan}, or, \textbf{quasiconformal} measure on the limit set of $\Gamma$.
	
	The case of Schottky groups was tackled in large generality in (unpublished as of July 28)\cite{garcía2023dimension} using the standard machinery of thermodynamic formalism. It seems reasonable to understand whether we can do this for the Bowen-Series coding for cocompact groups, but, so far, despite considerable advancements in understanding and applying the symbolic dynamics approach to hyperbolic groups (see \cite{ASENS_2013_4_46_1_131_0} or \cite{cantrell2022invariant}, for example), we still cannot extract the invariants for the harmonic measure. (author's remark: it seems that the harmonic measure resists the methods of thermodynamic formalism precisely because we do \textbf{not} expect it to maximize entropy, it doesn't seem to behave well with respect to any reasonable pressure operator...)
	
	\subsection{Higher-dimensional singularity conjecture}
	The conjecture can be formulated for discrete subgroups of $\text{Isom}_+(\mathbb{H}^n)$ for $n > 2$ as well. 
	
	\begin{theorem}[\cite{2019arXiv190411581R}]
		Let $\Gamma$ be a cofinite subgroup of $\text{Isom}_+(\mathbb{H}^n)$. Then, for any random walk on $\Gamma$ with finite $(n-1)$-st moment the hitting measure on $\partial \mathbb{H}^n$ is singular with respect to the Lebesgue measure.
	\end{theorem}
	\begin{theorem}[\cite{linaudpan}]
		Let $\Gamma$ be a subgroup of $\text{Isom}_+(\mathbb{H}^n)$ which does not contain a parabolic element. Then there always exists a random walk with a finite exponential moment and infinite support such that the harmonic measure is absolutely continuous with respect to the Patterson-Sullivan measure on the limit set.
	\end{theorem}
	
	The cocompact case is widely open, without a known example of a cocompact Kleinian group and a random walk on it admitting a singular hitting measure. The issue, is, for the most part, the lack of good repository of three- and higher-dimensional examples of Kleinian/reflection groups.
	
	\subsection{Higher-rank singularity conjecture}
	
	The conjecture can also be stated for discrete subgroups of $\text{SL}_n(\mathbb{R})$ as well.
	
	\begin{conjecture}[ \cite{MR2568439}, (\cite{kaimanovich2011matrix}, page 259)]
		\label{Higher rank singularity conjecture}
		For any finitely supported measure $\mu$ on $\text{SL}_n(\mathbb{R})$ for $n > 2$, whose support generates a discrete subgroup, the harmonic/hitting/stationary measure for the random walk driven by $\mu$ is singular with respect to the unique $O(n)$-invariant measure on the orbit $\partial S_a$, which is either the unique $O(n)$-invariant meausre itself, or its quotient if $a$ is non-simple.
	\end{conjecture}
	
	This version of the conjecture is widely considered to be tremendously more difficult than the rank-one version, for the following reasons.
	
	\begin{itemize}
		\item The first reason lies in the statement itself: we want to point the reader's attention to the fact that in the original paper it is not clearly stated which measure we should compare the hitting measure to. Moreover, we believe that the statement lacks sufficient motivation, as in rank 1 the Lebesgue class does contain the quasiconformal measures and the Patterson-Sullivan measures, and a rigidity result like Theorem \ref{BHMTheorem 1.4} does provide a powerful criterion for establishing the singularity of the harmonic measure.
		
		There seems to be no agreed upon satisfactory description in higher-rank. Nevertheless, we insist that G. Link's variation on the Patterson-Sullivan construction for higher-rank subgroups seems to be the closest to what we want, having the non-atomic property, a proper analogue of the Sullivan's shadow lemma, plus the absolute continuity with respect to the invariant measure on the flag space. See \cite{linkhausdorff} for more details.
		\item Continuing from the previous point, if we take the Kaimanovich-le Prince's statement as it is, we would want to assume that the limit set of $\Gamma$ contains the orbit $\partial S_a$, however, no satisfactory description of limit sets exists, akin to the rank 1 situation.
		\item Neither approach for rank-1 case immediately generalizes, as the Guivarch's homological approach fails for higher-dimensional manifolds, unless one replaces geodesics with higher-dimensional subsurfaces, but how does one relate (minimal?) surfaces to the random walk itself? And there is no version of Theorem \ref{BHMTheorem 1.4} for cocompact higher-rank groups either, as we lack the hyperbolic aspect, which allows every argument to go through in \cite{blachere2011harmonic}. It seems that the biggest obstacle lies in the non-existence of a good description of the Martin boundary for $(\Gamma, \mu)$, as we don't seem to have a suitable version of the Ancona's inequaltites due to the groups in question not being hyperbolic.
		\item In particular, we do not know if the equality of the Hausdorff dimensions is enough to claim absolute continuity. Moreover, it is not even clear how that the Hausdorff dimension is the \textbf{right} invariant to study: the most recent effort concentrates on the phenomenon of \textbf{exact-dimensionalty} of the hitting measures, see \cite{kaimanovich2011matrix}, \cite{lessaledrappier1}, \cite{ledrappier2023exact}. 
		\item Even if we had a proper rigidity result, one would have to know the geometric structure of the fundamental domains for uniform lattices, which also seems to be unknown. There are existing results for $\text{SL}_n(\mathbb{Z})$, see the thesis \cite{gisele}.
		\item Finally, higher-rank situation differs significantly from the rank-one setting because of the Margulis' rigidity theorems: in particular, we know that every lattice in $\text{SL}_n(\mathbb{R})$ has to be arthmetic. This suggests that there is no way one can resolve the conjecture at any level of generality without considering the arithmetic structure. On the contrary, one can argue that the rank-one methods are predominantly more geometric in nature.
	\end{itemize}
	
	\textbf{Exercise:} \cite{kaimanovich2011matrix} does claim that every countable Zariski-dense subgroup admits a measure $\mu$ such that the hitting measure with respect to $\mu$ is singular, by establishing the dimension drop. Write down such a measure explicitly for $\text{SL}_n(\mathbb{Z})$.
	
	
	
	\printbibliography
	
\end{document}